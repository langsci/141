% add all extra packages you need to load to this file  

%%%%%%%%%%%%%%%%%%%%%%%%%%%%%%%%%%%%%%%%%%%%%%%%%%%%
%%%                                              %%%
%%%           Examples                           %%%
%%%                                              %%%
%%%%%%%%%%%%%%%%%%%%%%%%%%%%%%%%%%%%%%%%%%%%%%%%%%%% 
%% to add additional information to the right of examples, uncomment the following line
% \usepackage{jambox}
%% if you want the source line of examples to be in italics, uncomment the following line
% \renewcommand{\exfont}{\itshape}
\usepackage{./langsci/styles/langsci-optional}
\usepackage{./langsci/styles/langsci-lgr}
\usepackage{./langsci/styles/langsci-glyphs}
\usepackage{tabularx} 

\usepackage[english]{babel}

\usepackage{tikz} %für Bäume
\usepackage{multirow}
\usepackage{longtable}%für Tabellen mit Seitenumbruch
\usepackage{colortbl} % für farbig hinterlegte Tabellenspalten
\newcommand\diag[4]{%
  \multicolumn{1}{p{#2}}{\hskip-\tabcolsep
  $\vcenter{\begin{tikzpicture}[baseline=0,anchor=south west,inner sep=#1]
  \path[use as bounding box] (0,0) rectangle (#2+2\tabcolsep,\baselineskip);
  \node[minimum width={#2+2\tabcolsep},minimum height=\baselineskip+\extrarowheight] (box) {};
  \draw (box.north west) -- (box.south east);
  \node[anchor=south west] at (box.south west) {#3};
  \node[anchor=north east] at (box.north east) {#4};
 \end{tikzpicture}}$\hskip-\tabcolsep}} %für diagonale Zeilen in tabellen
\usepackage{vowel} %for vowel chart
\usepackage{phonrule} %for formatting phonological rules
\usepackage{paralist} %for \compactitem
\usepackage{subfigure} %for subcaptions
\usepackage{gensymb} %for \degree
\usepackage[all]{xy} % for Figure 2.2 (noun class pairings)
\usepackage{multicol}
\usepackage{./langsci/styles/langsci-gb4e}