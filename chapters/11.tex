\chapter{Verbal nouns (infinitives)}
\label{ChapterInfinitives}\is{infinitive|(}
\section{Introduction}
In this chapter, verbal nouns (infinitives) will be discussed. After a description of their morphological structure and syntactic characteristics\is{syntax} (\sectref{VerbalNounsStructureCharacteristics}), the negation\is{negative} of verbal nouns and negative construction containing them will be described (\sectref{VerbalNounsNegation}). This is followed by an discussion of some of their functions (\sectref{VerbalNounsArguments}, \ref{VerbalNounsConverbsAndRelated}).
\section{Structure and characteristics of verbal nouns}\label{VerbalNounsStructureCharacteristics}
Verbal nouns (infinitives) share characteristics of both nouns and verbs. Formally, they are class 15 nouns\is{noun classes} and can hence be marked for one of the three \isi{locative} classes or carry the augment (\sectref{NominalMorphology}). Like any other noun phrase, infinitives can fulfil the syntactic\is{syntax} functions of subjects (\ref{exINFasSubject}), objects (\ref{exINFasObject}), the head of possessives (\ref{exINFasObject}), and of the dependent noun of the associative (\ref{exInfAssociative}).

\begin{exe}
\ex \label{exINFasSubject}
\gll kɪsita kʊ-bomb-a bo ʊ-lo, \textbf{kʊ}-ka-a-li=ko \textbf{ʊ}-\textbf{kʊ}-\textbf{keet}-\textbf{an}-\textbf{a} kʊ-maa-so n=ʊ-kʊ-ponani-a\\
without 15(\textsc{inf})-do-\textsc{fv} as \textsc{aug}-\textsc{ref}.11 15(\textsc{inf})-\textsc{neg}-\textsc{pst}-\textsc{cop}=17 \textsc{aug}-15(\textsc{inf})-see-\textsc{recp}-\textsc{fv} 17-6-eye \textsc{com}=\textsc{aug}-15(\textsc{inf})-greet.\textsc{recp}-\textsc{fv}\\
\glt `Without doing so, there was no looking each other in the eyes or greeting each other.' [Should she save a life\ldots]
\ex \label{exINFasObject} \gll n-\textbf{kʊ}-meenye \textbf{ʊ}-\textbf{kʊ}-\textbf{pɪɪj}-\textbf{a} \textbf{kw}-ake\\
\textsc{1sg}-15(\textsc{inf})-know.\textsc{pfv} \textsc{aug}-15(\textsc{inf})-cook-\textsc{fv} 15(\textsc{inf})-\textsc{poss.sg}\\
\glt \lq I know her cooking.' [ET]
\ex\label{exInfAssociative}
 \gll ba-a-tendekesy-aga ngatɪ kɪ-kombe ky-\textbf{a} \textbf{kʊ}-\textbf{nw}-\textbf{el}-\textbf{a} a-m-ɪɪsi\\
2-\textsc{pst}-prepare-\textsc{ipfv} like 7-cup 7-\textsc{assoc} 15(\textsc{inf})-drink-\textsc{appl}-\textsc{fv} \textsc{aug}-6-water\\
\glt \lq They prepared them [calabashes] just like a cup for drinking water.' [Lake Kyungululu]
\end{exe}

With respect to their verbal characteristics, verbal nouns can be modified by adverbials (\ref{exInfinitiveAdverbial}). They can take the complements licensed by the verb \isi{stem} and accordingly may carry an object marker;\is{object marker} see (\ref{exInfiniveImperfective}, \ref{exInfinitiveMotion1}) below. Further, infinitives can take post-final clitics;\is{enclitic} see (\ref{exNukuPreparationCulmination}) below.

\begin{exe}
\ex \label{exInfinitiveAdverbial}\gll ʊ-kʊ-jeng-a panandɪ$\sim$panandɪ jɪ-ka-j-a m-bombo n-gafu\\
\textsc{aug}-15-build-\textsc{fv} \textsc{redupl}$\sim$a\_little 9-\textsc{neg}-be(come)-\textsc{fv} 9-work 9-difficult\\ 
\glt `Building little by little is not difficult work.' [How to build modern houses]
\end{exe}

The \isi{stem} of verbal nouns consists of the base and the default final vowel -\textit{a}. With the movement\is{motion} grams (\textit{j})\textit{a} and \textit{isa} (\sectref{MovementGrams}), the infinitive may take the imperfective\is{aspect!imperfective} final suffix \mbox{-\textit{aga}}. The only other token of an infinitive carrying the imperfective\is{aspect!imperfective} suffix is the following example, where -\textit{aga} seems to indicate the generic aspect\is{aspect!generic} of the comitative infinitive vis-à-vis its perfective\is{aspect!perfective} superordinate verb.
\begin{exe}
\ex \label{exInfiniveImperfective} \gll a-ba-ndʊ ba-a-jeng-ile \textbf{n}=\textbf{ʊ}-\textbf{kʊ}-\textbf{tʊʊgasy}-\textbf{aga} n-ka-aja a-ko looli ba-a-taami-gw-aga ʊ-kʊ-ga-ag-a a-m-ɪɪsi a-g-a kʊ-nw-a n-ʊ-kʊ-nw-esy-a ɪ-mi-tiimo gy-abo\\ 
\textsc{aug}-2-person 2-\textsc{pst}-build-\textsc{pfv} \textsc{com}=\textsc{aug}-15-settle-\textsc{ipfv} 18-12-homestead \textsc{aug}-\textsc{ref.12} but 2-\textsc{pst}-trouble-\textsc{pass}-\textsc{ipfv} \textsc{aug}-15-6-find-\textsc{fv} \textsc{aug}-6-water \textsc{aug}-6-\textsc{assoc} 15-drink-\textsc{fv} \textsc{com}=\textsc{aug}-15-drink-\textsc{caus}-\textsc{fv} \textsc{aug}-4-herd 4-\textsc{poss.pl}\\
\glt 'People (had) built in that village but they had trouble finding water for drinking and watering their cattle.' [Water and toads]
\end{exe}
\section{Verbal nouns and negation}\label{VerbalNounsNegation}\is{negative|(}
Verbal nouns in Nyakyusa cannot be negated morphologically. To express the negation of an infinitive, periphrastic constructions are used. The most common one is (\textit{ʊ})\textit{kʊsita}, (\textit{ɪ})\textit{kɪsita} `without' followed by an augmentless infinitive (\ref{exKisita}). The former also figures in the negative counterpart to the \isi{narrative tense }(\sectref{NarrativeTense}).

\begin{exe}
\ex \label{exKisita} \gll lɪnga a-lɪ na=fyo a-bagiile ʊ-kʊ-bomb-a ɪ-m-bombo jo$\sim$j-oosa ɪ-jɪ i-kʊ-lond-a ʊ-kʊ-bomb-a \textbf{kɪsita} \textbf{kʊ}-\textbf{taami}-\textbf{gw}-\textbf{a}\\
if/when 1-\textsc{cop} \textsc{com}=\textsc{ref.8} 1-be\_able.\textsc{pfv} \textsc{aug}-15-work-\textsc{fv} \textsc{aug}-9-work \textsc{redupl}$\sim$9-all \textsc{aug}-\textsc{prox.9} 1-\textsc{prs}-want-\textsc{fv} \textsc{aug}-15-work-\textsc{fv} without 15-trouble-\textsc{pass}-\textsc{fv}\\
\glt \lq If he has them [tools], he can do any kind of work which he wants to do, without being bothered.' [Types of tools in the home]
\end{exe}

A construction for constituent negation consists of the substitutive as a \isi{proclitic} to the general negator \textit{mma}, followed by the infinitive carrying the augment.\footnote{Cf. also \citet[69]{SchumannK1899} and \citet[84f]{EndemannC1914}.}
\begin{exe}
\ex \label{exRefMma1}\gll kʊkʊtɪ ii-sikʊ i-kʊ-kʊ-tʊk-a, kangɪ i-kʊ-tɪ \textup{\lq\lq}ʊ-ne \textbf{ne}=\textbf{mma} \textbf{ʊ}-\textbf{kw}-\textbf{eg}-\textbf{igw}-\textbf{a} na Juma, n-ga-n̩-gan-a\textup{''}\\
every 5-day 1-\textsc{prs}-\textsc{2sg}-insult-\textsc{fv} again 1-\textsc{prs}-say \phantom{\lq\lq}\textsc{aug}-\textsc{1sg} \textsc{1sg}=no \textsc{aug}-15-marry-\textsc{pass}-\textsc{fv} \textsc{com} J. \textsc{1sg}-\textsc{neg}-1-love-\textsc{fv}\\
\glt `Every day she speaks badly about you and she says ``Me, I'm not getting married to Juma, I don't love him.''{}' [Juma, Asia and Sambuka]
\end{exe}

This construction, with the class 15 substitutive \textit{ko}, thus \textit{komma}, also serves to form negative commands (\ref{exKomma}). A free variant \textit{somma} is also found (\ref{exSomma1}, \ref{exSomma2}).\footnote{The source of the initial fricative is unclear.} These prohibitives can be adressed to a single person (\ref{exKomma}, \ref{exSomma1}) as well as to the second person plural (\ref{exSomma2}).
\begin{exe}
\ex \label{exKomma}\gll \textbf{komma} \textbf{ʊ}-\textbf{kʊ}-\textbf{nyonyw}-\textbf{a} ɪ-fi a-p-eeliigwe ʊ-n-nino\\
\textsc{proh} \textsc{aug}-15(\textsc{inf})-desire-\textsc{fv} \textsc{aug}-\textsc{prox}.8 1-give-\textsc{pass}.\textsc{pfv} \textsc{aug}-1-your\_companion\\
\glt `Do not desire what your neighbour has been given.' [Chief Kapyungu]
\ex \label{exSomma1}
\gll \textbf{somma} \textbf{ʊ}-\textbf{kʊ}-\textbf{paasy}-\textbf{a}! lee po keet-a, ʊ-ka-a-job-aga bo ʊ-kaalɪ ʊ-kʊʊ-ny-eeg-a?\\
\textsc{proh} \textsc{aug}-15-worry-\textsc{fv} now/but then watch-\textsc{fv} \textsc{2sg}-\textsc{neg}-\textsc{pst}-speak-\textsc{ipfv} as \textsc{2sg}-\textsc{pers} \textsc{aug}-15-\textsc{1sg}-take-\textsc{fv}?\\
\glt \lq Don't worry! Now look, why didn't you speak before picking me up?' [Crocodile and Monkey]
\ex \label{exSomma2}
\gll lɪnga m-b-iigal-iile \textbf{somma} \textbf{ʊ}-\textbf{kʊ}-\textbf{sook}-\textbf{a} pa-nja\\
if/when \textsc{1sg}-\textsc{2pl}-close-\textsc{appl.pfv} \textsc{proh} \textsc{aug}-15-leave-\textsc{fv} 16-outside\\
\glt \lq When I've locked you (pl.) in, don't go outside.' [Python and woman]
\end{exe}
\is{negative|)}
\section{Functions of verbal nouns}\label{VerbalNounsFunctions}

\subsection{Arguments of auxiliaries, modal and motion verbs}\label{VerbalNounsArguments}
Verbal nouns serve as complements of phasal verbs,\is{phasal verbs} also called \textit{aspectualizers}, such as \textit{anda} `begin, start', \textit{mala} `finish, stop', \textit{leka} `seize' or  \textit{endelela} `continue'. These take either the infinitive with the augment or the infinitive marked for \isi{locative} class 16 as their complement. The latter is illustrated in (\ref{exInfinitive16Aspectualizer}). For numerous examples of the first see Chapter \ref{AspectualClassification}. It is unclear how far the two differ in meaning and use. Speaker preferences seem to play a role: the younger language assistants used the class 16 form more frequently than the older assistants.

\begin{exe}
\ex \label{exInfinitive16Aspectualizer}
\gll i-kʊ-kwel-a kangɪ mu-m-piki n=ʊ-kw-\textbf{endelel}-a \textbf{pa}-\textbf{kw}-\textbf{ap}-\textbf{a} a-ma-peasi\\
1-\textsc{prs}-climb-\textsc{fv} again 18-3-tree \textsc{com}=\textsc{aug}-15-continue-\textsc{fv} 16-15-pick-\textsc{fv} \textsc{aug}-6-pear(<SWA)\\
\glt \lq He climbs up the tree again and continues to pick pears.' [Elisha pear story]
\end{exe}

Infinitives, either with the augment or marked for \isi{locative} class 16, also figure as arguments of \isi{modality} and manipulation verbs, where they alternate with the subjunctive\is{mood!subjunctive} (\sectref{SubjunctiveSubordinate}). The alternation between infinitives and the subjunctive mood\is{mood!subjunctive} is also found following predicative expressions of (dis-)approval or preference, including the invariants \textit{kyajɪpo} \lq (it is) better' and \textit{paakipo} \lq (it is) preferable'.

\begin{exe}
\ex \label{exInfinitiveApproval} \gll n̩-dʊ-baatɪko lw-a twe ba-Nyakyusa ʊ-n-nyambala \textbf{ʊ}-\textbf{kʊ}-\textbf{pɪɪj}-\textbf{a}, pamo \textbf{ʊ}-\textbf{kʊ}-\textbf{suk}-\textbf{a} ɪ-my-enda, pamo \textbf{ʊ}-\textbf{kʊ}-\textbf{neg}-\textbf{a} a-m-ɪɪsi bo ba-li=po a-ba-kiikʊlʊ \textbf{mw}-\textbf{iko}\\
18-11-procedure 11-\textsc{assoc} \textsc{1pl} 2-Ny. \textsc{aug}-1-man \textsc{aug}-15-cook-\textsc{fv} or \textsc{aug}-15-wash-\textsc{fv} \textsc{aug}-4-clothe or \textsc{aug}-15-draw\_liquid-\textsc{fv} \textsc{aug}-6-water as 2-\textsc{cop}=16 \textsc{aug}-2-woman 3-taboo\\
\glt \lq  In the custom of us, the Nyakyusa people, it is taboo for men to cook or wash clothes or draw water when women are around.' [Division of labour]
\end{exe}

Infinitives further function as oblique arguments of modal\is{modality} readings of verbs such as \textit{tola} \lq  defeat' and its passive \textit{toligwa} \lq fail', \textit{kɪnda} \lq surpass' or \textit{taamigwa} \lq be troubled' (\ref{exToligwa}). See also (\ref{exInfiniveImperfective}) on p.\nobreakspace\pageref{exInfiniveImperfective} above and (\ref{exOpeningHareTugutuSentence2}, \ref{exOpeningHareTugutuSentence3}) on p.\nobreakspace\pageref{exOpeningHareTugutuSentence2}.

\begin{exe}
\ex \label{exToligwa}
\gll tʊ-tol-iigwe ʊ-kʊ-lɪ-kol-a ii-bole\\
\textsc{1pl}-defeat-\textsc{pass.pfv} \textsc{aug}-15-5-grasp/hold-\textsc{fv} 5-leopard\\
\glt `We've failed to catch the leopard.' [Chief Kapyungu]
\end{exe}

Infinitives additionally marked for \isi{locative} classes\is{noun classes} 16 or 18 also constitute the lexical verb of periphrastic TMA constructions, namely the periphrastic progressive\is{aspect!progressive} (\sectref{Progressive}), the prospective/inceptive\is{aspect!prospective} (\sectref{ProspectiveKujapa}) and the \isi{narrative tense} (\sectref{NarrativeTense}). An infinitive with the augment or marked for one of the three \isi{locative} classes 16--18\is{noun classes} may further serve as the complement of the persistive\is{aspect!persistive} aspect \isi{auxiliary} (\sectref{Persistive}) and augmentless infinitives may serve as the semantic main verb of the movement grams (\sectref{MovementGrams}).\is{motion}

Lastly, verbs of \isi{motion} and related verbs such as \textit{ɪma} `stand, stop' or \textit{tʊma} \lq send' often take an infinitive complement additionally marked for one of the three \isi{locative} classes.\is{noun classes} Class 16 here indicates that the motion is in relation to a specific place where the eventuality of the verbal noun takes place (\ref{exLocInfMovementVerb16}). With class 17, this typically denotes motion with a purpose (\ref{exLocInfMovementVerb17purpose}). In a related fashion, a class 17 infinitive can specifically serve as a purpose clause in this context (\ref{exLocInfMovementVerb17both}). However, a pure motion reading \lq to / from' the eventuality is also possible (\ref{exLocInfMovementVerb17locational}). Infinitives marked for class 18 also predominantly give a purposive reading (\ref{exLocInfMovementVerb18purpose}), although a locational one is also attested (\ref{exLocInfMovementVerb18locational}).%\vspace{\baselineskip}
\begin{exe}
\ex \label{exLocInfMovementVerb16} \gll ɪ-m-bwa sy-ɪm-aga \textbf{pa}-\textbf{kʊ}-\textbf{ly}-\textbf{a} ɪ-fi-fupa\\
\textsc{aug}-10-dog 10-\textsc{pst}.stand/stop-\textsc{ipfv} 16-15-eat-\textsc{fv} \textsc{aug}-8-bone\\
\glt \lq The dogs would stop and eat the bones.' [Saliki and Hare]

\ex \label{exLocInfMovementVerb17purpose}
\gll a-ka-balɪlo ka-mo a-ba-hɪɪji ba-na ba-a-bʊʊk-ile \textbf{kʊ}-\textbf{kʊ}-\textbf{hɪɪj}-\textbf{a} ɪɪ-ng'ombe pa-kɪ-lo\\
\textsc{aug}-12-time 12-one \textsc{aug}-2-thief 2-four 2-\textsc{pst}-go-\textsc{pfv} 17-15-steal-\textsc{fv} \textsc{aug}-cow(10) 16-7-night\\
\glt \lq One time, four thieves went to steal cows at night.' [Wage of the thieves]
\ex \label{exLocInfMovementVerb17both}
\gll Kalʊlʊ a-lɪnkʊ-bʊʊk-a kʊ-lʊ-bʊbi \textbf{kʊ}-\textbf{kʊ}-\textbf{laalʊʊsy}-\textbf{a} lɪnga lʊ-mmw-ag-ile ʊ-n-kiikʊlʊ\\
Hare 1-\textsc{narr}-go-\textsc{fv} 17-11-spider 17-15-ask-\textsc{fv} if/when 11-1-find-\textsc{pfv} \textsc{aug}-1-woman\\
\glt \lq Hare went to Spider to ask if it had found a woman.' [Hare and Spider]
\ex \label{exLocInfMovementVerb17locational}
\gll bo lʊ-fum-ile \textbf{kʊ}-\textbf{kʊ}-\textbf{hah}-\textbf{a} \ldots\\
as 11-come\_from-\textsc{pfv} 17-15-seduce-\textsc{fv} {}\\
\glt \lq When it [Spider] returned from seducing \ldots' [Hare and Spider]
\ex \label{exLocInfMovementVerb18purpose}
\gll a-ba-ndʊ ba-a-bʊʊk-ile \textbf{n}-\textbf{kʊ}-\textbf{n}-\textbf{keet}-\textbf{a}\\
\textsc{aug}-2-person 2-\textsc{pst}-go-\textsc{pfv} 18-15-watch-\textsc{fv}\\
\glt \lq People went to see her.' [Mfyage turns into a lion]
\ex \label{exLocInfMovementVerb18locational}
\gll Saliki a-lɪnkw-is-a ʊ-kʊ-fum-a \textbf{n}-\textbf{kʊ}-\textbf{jaat}-\textbf{a}\\
S. 1-\textsc{narr}-come-\textsc{fv} \textsc{aug}-15-come\_from-\textsc{fv} 18-15-walk-\textsc{fv}\\
\glt \lq Saliki came from taking a walk.' [Saliki and Hare]
\end{exe}

\subsection{Uses as converbs and related functions}\label{VerbalNounsConverbsAndRelated}
Infinitives can be used in a fashion similar to converbs of simultaneity. The term converb is here understood in \citeauthor{HaspelmathM1995}'s (\citeyear[3]{HaspelmathM1995}) definition as \lq\lq a non-finite verb form whose main function is to mark adverbial subordination. Another way of putting it is that converbs are verbal adverbs''. Examples (\ref{exInfinitiveConverbSimultaneity1}--\ref{exInfinitiveMotion1}) illustrate this. The use of infinitives in a converb-like manner is especially common with verbs of motion, where each verb provides different components of a single motion event (\ref{exInfinitiveMotion1}).
\begin{exe}
\ex \label{exInfinitiveConverbSimultaneity1}\gll ba-lɪnkʊ-fimbɪlɪsy-a fiijo \textbf{ʊ}-\textbf{kʊ}-\textbf{n̩}-\textbf{daalʊʊsy}-\textbf{a} mpaka a-a-job-ile a-a-t-ile\\
2-\textsc{narr}-persuade-\textsc{fv} \textsc{intens} \textsc{aug}-15-1-ask-\textsc{fv} until 1-\textsc{pst}-speak-\textsc{pfv} 1-\textsc{pst}-say-\textsc{pfv}\\
\glt `They interrogated her much until she spoke.' [Killer woman]
\ex \label{exInfinitiveConverbSimultaneity2}\gll \textbf{ʊ}-\textbf{kʊ}-\textbf{keet}-\textbf{a} kʊ-mwanya ki-kʊ-bon-a ʊ-mu-ndʊ\\
\textsc{aug}-15-watch-\textsc{fv} 17-high 12-\textsc{prs}-see-\textsc{fv} \textsc{aug}-1-person\\
\glt \lq Looking up he sees a person.' [Nicholaus Pear Story]
\ex \label{exInfinitiveMotion1}\gll bo i-kw-and-a itolo ʊ-kʊ-kam-a, ʊ-n̩-dʊme a-lɪnkʊ-sook-a \textbf{ʊ}-\textbf{kʊ}-\textbf{fum}-\textbf{a} kʊʊ-sofu n=ʊ-kʊ-n-kol-a ʊ-n-kiikʊlʊ jʊ-la\\
as 1-\textsc{prs}-begin-\textsc{fv} just \textsc{aug}-15-milk-\textsc{fv} \textsc{aug}-1-husband 1-\textsc{narr}-leave-\textsc{fv} \textsc{aug}-15-come\_from-\textsc{fv} 17-room(9) \textsc{com}=\textsc{aug}-15-1-grasp/hold-\textsc{fv} \textsc{aug}-1-woman 1-\textsc{dist}\\
\glt `When she was starting to milk, the husband came out of [lit. left coming from] the bedroom and caught that woman.' [Killer woman]
\end{exe}

\label{ComitativeInfinitive}An infinitive together with a \isi{proclitic} form of the comitative \textit{na} can be used following another verb to create a tight link between the states-of-affairs of the two, which often occur in sequence. Most commonly the first verb is fully inflected. The relationship between these verbs can be one of cause and consequence (\ref{exNukuCauseConsequence}), preparation and culmination (\ref{exNukuPreparationCulmination}), or eventualities based on each other in a more general sense (\ref{exNukuBasedOnEachOther}). It is also attested with verbs expressing similar or conceptually related meanings (\ref{exNukuSimilarMeaning1}) and with the last verb in iconic repetitions (\ref{exNukuRepetition}). (\ref{exNukuINF}) illustrates coordination with a preceding infinitive complement.
\begin{exe}
\ex \label{exNukuCauseConsequence} \gll po jɪ-lɪnkʊ-jɪ-lʊm-a ɪɪ-nine \textbf{n}=\textbf{ʊ}-\textbf{kʊ}-\textbf{jɪ}-\textbf{gog}-\textbf{a}\\
then 9-\textsc{narr}-9-bite-\textsc{fv} \textsc{aug}-companion.9 \textsc{com}=\textsc{aug}-15-9-kill-\textsc{fv}\\
\glt `It [dog] bit the other one and killed it.' [Dogs laughed about each other]
\ex \label{exNukuPreparationCulmination} \gll i-kʊ-pɪmb-a ɪ-kɪ-kapʊ ky-osa n-ky-eni mu-n-jɪnga \textbf{n}=\textbf{ʊ}-\textbf{kʊ}-\textbf{sook}-\textbf{a}=\textbf{po}\\
1-\textsc{prs}-lift-\textsc{fv} \textsc{aug}-7-basket 7-all 18-7-forehead 18-9-bicycle \textsc{com}=\textsc{aug}-15-leave-\textsc{fv}=16\\
\glt `He loads a whole basket onto the front of his bicycle and rides away.' [Elisha Pear Story]
\ex \label{exNukuBasedOnEachOther}
\gll ʊ-n-kasi gw-a lʊ-bʊbi a-a-b-ambɪliile \textbf{n}=\textbf{ʊ}-\textbf{kʊ}-\textbf{ba}-\textbf{pɪɪj}-\textbf{ɪl}-\textbf{a} ɪ-fi-ndʊ ɪ-f-ingi fiijo\\
\textsc{aug}-1-wife 1-\textsc{assoc} 11-spider 1-\textsc{pst}-2-receive.\textsc{pfv} \textsc{com}=\textsc{aug}-15-2-cook-\textsc{appl}-\textsc{fv} \textsc{aug}-8-food \textsc{aug}-8-many \textsc{intens}\\
\glt `Spider's wife received them and cooked a lot of food for them.' [Hare and Spider]
\ex \label{exNukuSimilarMeaning1}\gll n̩goosi a-lɪnkʊ-kʊl-a \textbf{n}=\textbf{ʊ}-\textbf{kʊ}-\textbf{kiikʊlʊp}-\textbf{a}\\
N. 1-\textsc{narr}-grow-\textsc{fv} \textsc{com}=\textsc{aug}-15-become\_woman-\textsc{fv}\\
\glt `Ngoosi grew up and became a woman.' [Man and his in-law]
\ex\label{exNukuRepetition} \gll boo=bʊno$\sim$bʊ-no ba-lɪnkw-end-a, ba-lɪnkw-end-a, ba-lɪnkw-end-a \textbf{n}=\textbf{ʊ}-\textbf{kw}-\textbf{end}-\textbf{a}\\
\textsc{ref.14}=\textsc{redupl}$\sim$14-\textsc{dem} 2-\textsc{narr}-walk/travel-\textsc{fv} 2-\textsc{narr}-walk/travel-\textsc{fv} 2-\textsc{narr}-walk/travel-\textsc{fv} \textsc{com}=\textsc{aug}-15-walk/travel-\textsc{fv}\\
\glt \lq Thus they travelled, travelled, travelled and travelled.' [Pregnant women]
% bsp leider doppelt verwendet
\ex\label{exNukuINF} \gll ɪ-n-gwina j-iis-aga n-kʊ-j-eeg-a ɪ-n-gambɪlɪ \textbf{n}=\textbf{ʊ}-\textbf{kʊ}-\textbf{bʊʊk}-\textbf{a} na=jo pa-lʊ-sʊngo pa-kw-angal-a\\
\textsc{aug}-9-crocodile 9-\textsc{pst}.come-\textsc{ipfv} 18-15-9-take-\textsc{fv} \textsc{aug}-9-monkey \textsc{com}=\textsc{aug}-15-go-\textsc{fv} \textsc{com}=\textsc{ref.9} 16-11-island 16-15-be\_well-\textsc{fv}\\
\glt \lq Crocodile used to come to pick up monkey and go with him to an island to spend time together.' [Crocodile and Monkey] %beispiel schon bei NARR
\end{exe}

This structure is conventionalized with the verb \textit{enda} \lq walk/travel' as the first verb and serves as a marker of sequential events:

\begin{exe}
\ex \gll ʊ-n-hɪɪj-i ʊ-jʊ a-a-longwile n-ky-eni a-lɪnkw-\textbf{end}-a \textbf{n}=\textbf{ʊ}-\textbf{kʊ}-\textbf{kol}-\textbf{a} ʊ-lw-igi\\
\textsc{aug}-1-thief \textsc{aug}-\textsc{prox.1} 1-\textsc{pst}-lead.\textsc{pfv} 18-7-forehead 1-\textsc{narr}-walk/travel-\textsc{fv} \textsc{com}=\textsc{aug}-15-grasp/hold-\textsc{fv} \textsc{aug}-11-door\\
\glt \lq The thief who was going ahead then grabbed the door.' [Wage of the thieves]
\end{exe}

Most commonly, only one verb in a sequence is expressed by the comitative infinitive. In a few cases, however, up to three verbs (\ref{exNuku3verbs}) marked in this manner can be found.
\begin{exe}
\ex \label{exNuku3verbs}\gll a-a-gomok-a ʊ-mw-anike jʊ-la \textbf{n}=\textbf{ʊ}-\textbf{kʊ}-\textbf{fik}-\textbf{a} pa-ka-aja pa-la \textbf{n}=\textbf{ʊ}-\textbf{kʊ}-\textbf{m̩}-\textbf{bʊʊl}-\textbf{a} \textbf{n}=\textbf{ʊ}-\textbf{kʊ}-\textbf{n}-\textbf{nangɪsy}-\textbf{a} ɪ-si ʊ-n̩-dʊme\\
1-\textsc{subsec}-return-\textsc{fv} \textsc{aug}-1-young\_person 1-\textsc{dist} \textsc{com}=\textsc{aug}-15-arrive-\textsc{fv} 16-12-homestead 16-\textsc{dist} \textsc{com}=\textsc{aug}-15-1-tell-\textsc{fv} \textsc{com}=\textsc{aug}-15-1-show-\textsc{fv} \textsc{aug}-\textsc{prox.10} \textsc{aug}-1-husband\\
\glt `That young woman returned and arrived at home and told and showed her husband these things.' [Man and his in-law]
\end{exe}

Other infinitives serve a variety of functions which likely go back to their converb-like use. The infinitive of \textit{tɪ} \lq say' amongst other things serves as a complementizer; see \sectref{defectiveti}. The reciprocal/associative\is{reciprocal} of \textit{konga} `follow' is used as an infinitive in a preposition-like manner, together with a comitative phrase expressing reason (\ref{exINFkongana}). Similarly the infinitive of \textit{fika} \lq arrive', \textit{fuma} \lq come from', and \textit{anda} \lq start', as well as its \isi{applicative} \textit{andɪla}, are used in a preposition-like manner. Note that in the case of the first two, the original spatial meaning has been extended to a temporal one. (\ref{exINFfuma}, \ref{exINFandila}) illustrate this use for \textit{fuma} and \textit{andɪla}.
\begin{exe}
\ex \label{exINFkongana}\gll nalooli \textbf{ʊ}-\textbf{kʊ}-\textbf{kong}-\textbf{an}-\textbf{a} \textbf{n}=ʊ-lʊ-gano ʊ-lʊ a-a-lɪ na=lo n̩goosi a-lɪnkʊ-jong-a n=ʊ-n-nyambala jʊ-mo\\
really \textsc{aug}-15-follow-\textsc{recp}-\textsc{fv} \textsc{com}=\textsc{aug}-11-love \textsc{aug}-\textsc{prox.11} 1-\textsc{pst}-\textsc{cop} \textsc{com}=\textsc{ref.11} N. 1-\textsc{narr}-run\_away-\textsc{fv} \textsc{com}=\textsc{aug}-1-man 1-one\\
\glt `Because of the love that Ngoosi had, she eloped with a man.' [Man and his in-law]
\ex \label{exINFfuma}
\gll ʊ-mu-ndʊ ʊ-jʊ a-fumwike \textbf{ʊ}-\textbf{kʊ}-\textbf{fum}-\textbf{a} pa-tali\\
\textsc{aug}-1-person \textsc{aug}-\textsc{prox.1} 1-be(come)\_famous.\textsc{pfv} \textsc{aug}-15-come\_from-\textsc{fv} 16-long\\
\glt \lq This person has been famous since long ago.' [ET]
\ex \label{exINFandila}
\gll tw-al-iiswisye ɪ-mi-fuko \textbf{ʊ}-\textbf{kw}-\textbf{and}-\textbf{ɪl}-\textbf{a} n=ʊ-lʊ-bʊnjo mpaka pa-muu-si\\
\textsc{1pl}-\textsc{pst}-be(come)\_full.\textsc{caus.pfv} \textsc{aug}-4-sack \textsc{aug}-15-begin-\textsc{appl}-\textsc{fv} \textsc{com}=\textsc{aug}-11-morning until 16-3-daytime\\
\glt \lq We filled sacks from morning till afternoon.' [ET]
\end{exe}%

Lastly, the infinitive of \textit{anda} `begin, start' is further used as the dependent noun in the associative construction as the ordinal number `first':

\begin{exe}
\ex \gll a-lɪnkw-is-a ʊ-mu-ndʊ ʊ-gw-a kw-and-a\\
1-\textsc{narr}-come-\textsc{fv} \textsc{aug}-1-person \textsc{aug}-1-\textsc{assoc} 15-begin-\textsc{fv}\\
\glt `The first person came.' [Chief Kapyungu]
\end{exe}
\is{infinitive|)}