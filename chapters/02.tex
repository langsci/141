\chapter{Grammatical sketch}
\section{Typological overview}
Nyakyusa classifies as a typical Narrow Bantu language, while also showing some unique characteristics. This section will provide a summary of these characteristics, giving examples of a number of commonly mentioned defining features (\citealt{MoehligW1981}; \citealt{NurseDPhillipsonG2003a}). It will then provide a short grammatical sketch that will serve as a point of departure for a more detailed description of the verb.

The phonology of Nyakyusa proves to be typical for Narrow Bantu in terms of its symmetrical inventory of seven vowel qualities\is{vowels!inventory} and its phonotactics, although it can be considered innovative with regard to the loss of \isi{tone} (\sectref{Stress}). Nyakyusa also shows a rather particular variation from typical Narrow Bantu vowel harmony\is{vowels!vowel height harmony} and related processes (\sectref{MorphophonologyOfVerbalExtension}).

Concerning morphology, Nyakyusa is a highly agglutinative language, with a productive system of verbal derivation.\is{derivation} As will be described in \sectref{Negations}, it features various verbal negations,\is{negative} the functional distribution of which is uncommon from a pan-Bantu perspective. Nyakyusa further has an elaborate system of tense,\is{tense} aspect\is{aspect!grammatical} and \isi{mood} categories. An outstanding feature of Nyakyusa in this regard are the great number of \isi{futurate} constructions (Chapter \ref{Futurates}) in comparison to past tense\is{tense!past} forms, and the existence of two different \isi{narrative markers} (Chapter \ref{NarrativeMarkers}). Bantu morphology, especially derivational\is{derivation} morphology, has been described as showing a strong tendency towards \isi{templatic} structures \citep{HymanL2002}. Nyakyusa shows some intriguing cases of \isi{templatic} requirements in the phonological realisation of certain affixes and affix combinations (see \sectref{ApplicativizedCausatives}, \ref{Imbrication}) that are noteworthy even within the broader Bantu context.

When it comes to syntax,\is{syntax} the Nyakyusa \lq\lq basic'' word order is SVO. The language has compound verbs, that is auxiliaries\is{auxiliary} taking an inflected verb as a complement (see \sectref{Persistive}, \ref{MinorConstructions}). The language further has a well developed system of \isi{noun classes} and both subject-\is{subject marker} and object-marking\is{object marker} on the verb.

\section{Basic phonology}\label{BasicPhonology}
\subsection{Vowels and syllables}\label{VowelsAndSyllales} 
\subsubsection{Syllable structure}
Syllables in Nyakyusa can have any of the following structures:
\begin{exe}
		\ex
		\begin{tabbing}
			NCGVx\=x\textit{ʊ.lʊ.te.\textbf{ngaa}.no}x\=\kill%unnsinnszeile für tabulatoren
			V \> \textit{\textbf{ɪ}.kɪ.ko.ta}\> `chair' \\
			VV \> \textit{\textbf{ii}.si.kʊ}\> `day' \\
			CV \> \textit{ʊ.n.jwe.\textbf{go}}\> `noise' \\
			CVV \> \textit{a.\textbf{kaa}.ja}\> `homestead' \\
			CGV \> \textit{ɪ.\textbf{kya}.lo}\> `field' \\
			N\> \textit{ʊ.\textbf{m}.pʊ.nga}\> `rice'\\
			NCV \> \textit{ʊ.mu.\textbf{ndʊ}}\> `person' \\
			NCVV \> \textit{ʊ.lʊ.te.\textbf{ngaa}.no}\> `peace' \\		
			NCGV \> \textit{ɪ.\textbf{ngwe}.go}\> `spear'
		\end{tabbing}
\end{exe}

The possible syllable structures are subject to several distributional constraints: Syllables featuring an initial NC-cluster are not permitted following a syllabic nasal and syllabic nasals do not occur in word-final position. V and VV only occur in the initial position of a word or clitic group. Word-final vowels as a general rule are short, although there are several exceptions: the negated copula (optional, see \sectref{Copulae}), some cases of the associative marker -\textit{a}, some interjections such as \textit{ee} `yes' and a number of ideophones.\is{ideophones}

\subsubsection{Vowel inventory}
 \is{vowels!inventory|(}
Nyakyusa has a system of seven phonemic vowel qualities, as has also been reconstructed for \ili{Proto-Bantu} \citep[147]{SchadebergT2003b}. Phonetically, the mid-vowels are realized as open-mid vowels [ɛ, ɔ]. All seven vowel qualities occur as short and long.  Following Bantuist conventions, long vowels are represented by a sequence of two identical vowels.\footnote{This stands in contrast to Ndali,\il{Ndali} which has a reduced inventory of five phonemic vowel qualities, maintaining the inherited length opposition (\citealt{LabroussiC1998}; \citealt{BotneR2008}), and Ngonde,\il{Ngonde} which has further lost the opposition in quantity (\citealt{LabroussiC1998}; \citealt{KishindoP1999}).} \figref{exVokalsystem} illustrates the vowel inventory.
%\begin{exe}[h]
%\ex\label{exVokalsystem}
\begin{figure}[h]
	\begin{center}
		\begin{vowel}[plain,triangle]%Vokaldreiecke
			\putvowel{a}{2\vowelhunit}{3\vowelvunit}
			\putvowel{i}{0pt}{0pt}
			\putvowel{ɪ}{0,67\vowelhunit}{1\vowelvunit}
			\putvowel{e}{1,34\vowelhunit}{2\vowelvunit}
			\putvowel{o}{2,67\vowelhunit}{2\vowelvunit}
			\putvowel{ʊ}{3,34\vowelhunit}{1\vowelvunit}
			\putvowel{u}{4\vowelhunit}{0pt}
		\end{vowel}
		\begin{vowel}[plain,triangle]
			\putvowel{aa}{2\vowelhunit}{3\vowelvunit}
			\putvowel{ii}{0pt}{0pt}
			\putvowel{ɪɪ}{0,67\vowelhunit}{1\vowelvunit}
			\putvowel{ee}{1,34\vowelhunit}{2\vowelvunit}
			\putvowel{oo}{2,67\vowelhunit}{2\vowelvunit}
			\putvowel{ʊʊ}{3,34\vowelhunit}{1\vowelvunit}
			\putvowel{uu}{4\vowelhunit}{0pt}
		\end{vowel}
	\end{center}
\caption{Nyakyusa vowel system}\label{exVokalsystem}
\end{figure}


\subsubsection{Vowel length}\label{VowelLength}  \is{vowels!length|(}
Nyakyusa has both short and long vowels. They can be lexically specified or can arise through morphophonemic processes such as contact between heteromorphemic vowels (\sectref{HiatusSolution}) or as a compensation for the deletion of word-internal non-syllabic nasals (see p.\nobreakspace\pageref{NasalDelition} in \sectref{NasalDelition}). It is these contexts in which vowel length is distinctive. The minimal pairs in (\ref{exMinimalpaareVokLängeLex}) exemplify the distinctive lexical function of vowel length, while those in (\ref{exMinimalpaareVokLängeGramm}) show that vowel length is grammatically distinctive.
\begin{exe}
	\ex \label{exMinimalpaareVokLängeLex}
	\begin{tabbing}
		\textit{ʊkʊmanya}x\=`err, make a mistake'x\=vs.x\=\textit{ʊkʊʊmanya}x\=\kill
		\textit{ʊkʊtima}\>`to rain'\>vs.\>\textit{ʊkʊtiima}\>`to herd'
		\\\textit{ʊkʊlɪla}\>`to cry'\>vs.\>\textit{ʊkʊlɪɪla}\>`to eat with/for/at'
		\\\textit{ʊkʊmeta}\>`cut (esp. hair)'\>vs.\>\textit{ʊkʊmeeta}\>`bleat, cry like a goat'
		\\\textit{ʊkʊsala}\>`choose; pick'\>vs.\>\textit{ʊkʊsaala}\>`be(come) happy'
		\\\textit{ʊkʊbola}\>`rot, go bad'\>vs.\>\textit{ʊkʊboola}\>`cut; slaughter'
		\\\textit{ʊkʊtʊla}\>`err, make a mistake'\>vs. \>\textit{ʊkʊtʊʊla}\>`take from the head'
		\\\textit{ʊkʊfula}\>`castrate'\>vs.\>\textit{ʊkʊfuula}\>`undress; open knots'		
	\end{tabbing}	
	\ex \label{exMinimalpaareVokLängeGramm}
	\begin{tabbing}
		\textit{ʊkʊmanya}x\=`err, make a mistake'x\=vs.x\=\textit{ʊkʊʊmanya}x\=\kill
		\textit{ajengile}\>`s/he has built'\>vs.\>\textit{aajengile}\>`s/he built'\\
		\textit{ʊkʊmanya}\>`to know'\>vs.\>\textit{ʊkʊʊmanya}\>`to know me' \\
		\textit{ʊkʊsala}\>`to choose; pick'\>vs.\>\textit{ʊkʊʊsala}\>`to choose me'
	\end{tabbing}	
\end{exe}
A number of Bantu languages, including some of the neighbouring Corridor languages, restrict the occurrence of long vowels. For instance, in \ili{Malila} M24, a long vowel cannot surpass the fourth mora reckoned from the word end \citep{KutschLojengaC2007}. Nyakyusa, however, does not know such a constraint as the following examples illustrate.
\begin{exe}
	\ex
	\begin{tabular}[t]{@{}>{\itshape}ll}
		\textit{j\textbf{aa}katala} & \lq (then) it (class 9) got angry'\\
		\textit{\textbf{aa}l\textbf{aa}l\textbf{ʊʊ}siisye} & `s/he asked'\\
		\textit{b\textbf{aa}lʊʊki} & `oranges'
	\end{tabular}
\end{exe}
Long vowels can further be the result of compensatory lengthening before prenasalized plosives (see \sectref{NC-Clusters}). Vowels following glides (\sectref{glides}) are also realized with a slightly increased duration. As vowel length in these two environments is predictable, it is not given in the orthographic representation.
Generally speaking, vowels in a syllable bearing primary stress tend to be realized with a longer phonetic duration, especially in the penultimate syllable of a phrase. Nevertheless, the phonemic length distinction is still discernible in stressed syllables.\footnote{The accoustic study by \citet{vanEssenOKaehler-MeyerE1969} notes the same for nouns produced in citation form, hence a phonological phrase. Such cases of phonetic penultimate lengthening have been reported for a number of Bantu languages; see \citet[313]{HymanL2013} and references therein.} The only exception to the regular lengthening before prenasalized plosives is in imperatives (\sectref{Imperative}).

The opposite of compensatory lengthening is triggered by syllabic nasals (see \sectref{SyllabicNasals}).\footnote{The same effect is triggered by the object marker of noun class 1; see p.\nobreakspace\pageref{OMNCL1syllabic} in \sectref{OMNCL1syllabic}.} Any vowel preceding a syllabic nasal surfaces as short. The examples in (\ref{exNCvsSyllabicNasal}) illustrate the opposition between prenasalized plosives and syllabic nasals. In the practical orthography employed throughout this work, syllabic nasals are marked <n̩, m̩> preceding <b, d, g>, thus differentiating them from the voiced prenasalized plosives. As a nasal preceding a voiceless plosive or another nasal is always syllabic in Nyakyusa  -- see \sectref{SyllabicNasals}, \ref{PrenasalizedPlosives} -- syllabicity is not overtly indicated in these cases. Examples in (\ref{exVowelShortening}) show that a short vowel is pronounced where the outcome of morphophonology would otherwise be a long or lengthened vowel.

\begin{exe}
	\ex
	\begin{xlist}
		\sn
		\label{exNCvsSyllabicNasal}
		\begin{tabbing}
			{[}a.ŋ̩.ga.ˈni.ɾɛ{]}x\=`s/he loves him/her'x\=vs.x\=\kill
			{[}a.ŋ̩.ga.ˈni.ɾɛ{]}\>`s/he loves him/her\>vs.\>{[}aː.{\ᵑ}ga.ˈni.ɾɛ{]} `s/he loves me'
		\end{tabbing}
	\end{xlist}
	\ex \label{exVowelShortening} 
	\begin{xlist}
		\ex Long vowel (triggered by \textsc{1sg} prefix) shortened:
		\begin{tabbing}
			{[tʷa.ŋ̩.kʰɔ.ˈma.ɰa]}x\=(\degree ba-a-mu-kom-aga)x\=\kill
			{[na.ŋ̩.kʰɔ.ˈma.ɰa]}\>(\degree n-a-mu-kom-aga)\>`I was beating him/her'
		\end{tabbing}
		\ex Long vowel (outcome of vowel coalescence) shortened:
		\begin{tabbing}
			{[tʷa.ŋ̩.kʰɔ.ˈma.ɰa]}x\=(\degree ba-a-mu-kom-aga)x\=\kill
			{[β̞a.ŋ̩.kʰɔ.ˈma.ɰa]}\>(\degree ba-a-mu-kom-aga)\>`they were beating him/her'
		\end{tabbing}
		\ex Vowel shortening overrides compensatory lengthening:
		\begin{tabbing}
			{[tʷa.ŋ̩.kʰɔ.ˈma.ɰa]}x\=(\degree ba-a-mu-kom-aga)x\=\kill
			{[tʷa.ŋ̩.kʰɔ.ˈma.ɰa]}\>(\degree tʊ-a-mu-kom-aga)\>`we were beating him/her'
		\end{tabbing}
	\end{xlist}
\end{exe}
\is{vowels!length|)}\is{vowels!inventory|)}

\subsubsection{Vowel coalescence and glide formation}\label{HiatusSolution}\is{vowels!hiatus solution|(}
When heteromorphemic vowels within a word are juxtaposed, vowel coalescence and glide formation take place. The regularities differ slightly to the left and to the right of the root. (\ref{exVowelContacti}--\ref{exVowelContactu}) illustrate the outcome to the left of the stem, using the combination of subject prefix and verb stem. In the case of /u/-initial stems, the examples are nouns.\footnote{No verb stems beginning with /u/ are attested and no prefix contains the mid-vowels /e, o/; see \sectref{PhonologicalStructureVerbalMorphemes}.} Note that the outcome of vowel contact is independent of syllable count.

\clearpage % necessary to keep following examples together
\begin{exe}
	\ex 
	\label{exVowelContacti}
	\begin{tabbing}
		u + ax\=$\rightarrow$x\=waxx\=\textit{ʊmuunyu}x\=(\degree ʊ-mu-unyu)x\=\kill%Unsinnszeile für Tabulatoren
		i + i\>$\rightarrow$\>ii\>\textit{siisile}\>(\degree si-is-ile)\>`they (class 10) have come'\\
		i + ɪ\>$\rightarrow$\>yɪ\>\textit{syɪmile}\>(\degree si-ɪm-ile)\>`they (class 10) have stopped'\\
		i + e\>$\rightarrow$\>ye\>\textit{syegile}\>(\degree si-eg-ile)\>`they (class 10) have taken'\\
		i + a\>$\rightarrow$\>ya\>\textit{syagile}\>(\degree si-ag-ile)\>`they (class 10) have found'\\
		i + o\>$\rightarrow$\>yo\>\textit{syogile}\>(\degree si-og-ile)\>`they (class 10) have bathed'\\
		i + ʊ \>$\rightarrow$\>yʊ\>\textit{syʊmile}\>(\degree si-ʊm-ile)\>`they (class 10) have dried'\\ 
		i + u\>$\rightarrow$\>yu\>\textit{ɪfyuga}\>(\degree ɪ-fi-uga)\>`hoofs'
	\end{tabbing}
	\ex
	\begin{tabbing}
		u + ax\=$\rightarrow$x\=waxx\=\textit{ʊmuunyu}x\=(\degree ʊ-mu-unyu)x\=\kill%Unsinnszeile für Tabulatoren
		ɪ + i\>$\rightarrow$\>ii\>\textit{liisile}\>(\degree lɪ-is-ile)\>`it (class 5) has come'\\
		%\>\>\>\textit{ɪkiisʊ}\>(\degree ɪ-kɪ-isʊ)\>`land; country'\\
		ɪ + ɪ\>$\rightarrow$\>yɪ\>\textit{lyɪmile}\>(\degree lɪ-ɪm-ile)\>`it (class 5) has stopped'\\
		%\>\>\>\textit{ɪkyɪnja}\>(\degree ɪ-kɪ-ɪnja)\>`year'\\
		ɪ + e\>$\rightarrow$\>ye\>\textit{lyegile}\>(\degree lɪ-eg-ile)\>`it (class 5) has taken'\\
		%\>\>\>\textit{ɪkyeni}\>(\degree ɪ-kɪ-eni)\>`forehead'\\
		ɪ + a\>$\rightarrow$\>ya\>\textit{lyagile}\>(\degree lɪ-ag-ile)\>`it (class 5) has found'\\
		%\>\>\>\textit{ɪkyalo}\>(\degree ɪ-kɪ-alo)\>`farm'\\
		ɪ + o\>$\rightarrow$\>yo\>\textit{lyogile}\>(\degree lɪ-og-ile)\>`it (class 5) has bathed'\\
		%\>\>\>\textit{ɪkyolo}\>(\degree ɪ-fy-ola)\>`pouch, bag'\\
		ɪ + ʊ\>$\rightarrow$\>yʊ\>\textit{lyʊmile}\>(\degree lɪ-ʊm-ile)\>`it (class 5) has dried'\\
		%\>\>\>\textit{ɪkyʊma}\>(\degree ɪ-kɪ-ʊma)\>`property; riches'\\
		ɪ + u \>$\rightarrow$\>yu\>\textit{ɪkyuga}\>(\degree ɪ-kɪ-uga)\>`hoof'
	\end{tabbing}
	\ex
	\begin{tabbing}
		u + ax\=$\rightarrow$x\=waxx\=\textit{ʊmuunyu}x\=(\degree ʊ-mu-unyu)x\=\kill%Unsinnszeile für Tabulatoren
		a + i\>$\rightarrow$\>ii\>\textit{biisile}\>(\degree ba-is-ile)\>`they have come'\\
		%\>\>\>\textit{amiino}\>(\degree a-ma-ino)\>`teeth'\\
		a + ɪ\>$\rightarrow$\>ɪɪ\>\textit{bɪɪmile}\>(\degree ba-ɪm-ile)\>`they have stopped'\\
		%\>\>\>\textit{amɪɪsi}\>(\degree a-ma-ɪsi)\>`water'\\
		a + e\>$\rightarrow$\>ee\>\textit{beegile}\>(\degree ba-eg-ile)\>`they have taken'\\
		%\>\>\>\textit{ameebe}\>(\degree a-ma-ebe)\>`crows'\\
		a + a\>$\rightarrow$\>aa\>\textit{baagile}\>(\degree ba-ag-ile)\>`they have found'\\
		%\>\>\>\textit{amaani}\>(\degree a-ma-ani)\>`leafs'\\
		a + o\>$\rightarrow$\>oo\>\textit{boogile}\>(\degree ba-og-ile)\>`they have bathed'\\
		%\>\>\>\textit{amoobe}\>(\degree a-ma-obe)\>`fingers, toes'\\  check if semantically OK
		a + ʊ\>$\rightarrow$\>ʊʊ\>\textit{bʊʊmile}\>(\degree ba-ʊm-ile)\>`they have dried'\\
		a + u\>$\rightarrow$\>uu\>\textit{amungu}\>(\degree a-ma-ungu)\>`mushrooms'
	\end{tabbing}
	\ex
	\begin{tabbing}
		u + ax\=$\rightarrow$x\=waxx\=\textit{ʊmuunyu}x\=(\degree ʊ-mu-unyu)x\=\kill%Unsinnszeile für Tabulatoren
		ʊ + i\>$\rightarrow$\>wi\>\textit{twisile}\>(\degree tʊ-is-ile)\>`we have come'\\
		%\>\>\>\textit{ʊlwiho}\>(\degree ʊ-lw-iho)\>`law; custom; culture'\\
		ʊ + ɪ\>$\rightarrow$\>wɪ\>\textit{twɪmile}\>(\degree tʊ-ɪm-ile)\>`we have stopped'\\
		%\>\>\>\textit{ʊlwɪsi}\>(\degree ʊ-lʊ-ɪsi)\>`river'\\
		ʊ + e\>$\rightarrow$\>we\>\textit{twegile}\>(\degree tʊ-eg-ile)\>`we have taken'\\
		%\>\>\>\textit{ʊlwelo}\>(\degree ʊ-lʊ-elo)\>`fishing net'\\
		ʊ + a\>$\rightarrow$\>wa\>\textit{twagile}\>(\degree tʊ-ag-ile)\>`we have found'\\
		%\>\>\>\textit{ʊlwalo}\>(\degree ʊ-lʊ-alo)\>`base, foundation'\\
		ʊ + o\>$\rightarrow$\>oo\>\textit{toogile}\>(\degree tʊ-og-ile)\>`we have bathed' \\
		ʊ + ʊ\>$\rightarrow$\>ʊʊ\>\textit{tʊʊmile}\>(\degree tʊ-ʊm-ile)\>`we have dried'\\
		%\>\>\>\textit{ʊlʊʊbo}\>(\degree ʊ-lʊ-ʊbo)\>`big knife, sword'\\
		ʊ + u\>$\rightarrow$\>uu\>\textit{ʊtungu}\>(\degree ʊ-tʊ-ungu) \> \lq little mushrooms' %check if OK
	\end{tabbing}
	\ex\label{exVowelContactu}\begin{tabbing}
		u + ax\=$\rightarrow$x\=waxx\=\textit{ʊmuunyu}x\=(\degree ʊ-mu-unyu)x\=\kill%Unsinnszeile für Tabulatoren
		u + i\>$\rightarrow$\>wi\>\textit{mwisile}\>(\degree mu-is-ile)\>`you (pl.) have come'\\
		%\>\>\>\textit{ʊmwifwa}\>(\degree ʊ-mu-ifwa)\>`thorn'\\
		u + ɪ\>$\rightarrow$\>wɪ\>\textit{mwɪmile}\>(\degree mu-ɪm-ile)\>`you (pl.) have stopped'\\
		%\>\>\>\textit{ʊmwɪmbi}\>(\degree ʊ-mu-ɪmbi)\>`singer'\\
		u + e\>$\rightarrow$\>we\>\textit{mwegile}\>(\degree mu-eg-ile)\>`you (pl.) have taken'\\
		%\>\>\>\textit{ʊmwesi}\>(\degree ʊ-mu-esi)\>`moon; month'\\
		u + a\>$\rightarrow$\>wa\>\textit{mwagile}\>(\degree mu-ag-ile)\>`you (pl.) have found'\\
		%\>\>\>\textit{ʊmwafi}\>(\degree ʊ-mw-afi)\>`medicine to make sb. vomit'\\
		u + o\>$\rightarrow$\>oo\>\textit{moogile}\>(\degree mu-og-ile)\>`you (pl.) have bathed' \\
		%\>\>\>\textit{ʊmooto}\>(\degree ʊ-mu-oto)\>`fire'\\
		u + ʊ\>$\rightarrow$\>ʊʊ\>\textit{mʊʊmile}\>(\degree mu-ʊm-ile)\>`you (pl.) have dried'\\
		%\>\>\>\textit{ʊmʊʊji}\>(\degree ʊ-mu-ʊji)\>`breath; air; steam'\\ 
		u + u\>$\rightarrow$\>uu\>\textit{ʊmuunyu}\>(\degree ʊ-mu-unyu)\>`salt'
	\end{tabbing}
\end{exe}

The table in (\ref{exVowelContactPrefixingTable}) summarizes the outcome of left-of-the-stem (prefixing) vowel contact.

%\begin{table}
%\centering
%\captionabove{Adjacent vowels (prefixing):}
\begin{exe}\ex\label{exVowelContactPrefixingTable}
	Adjacent vowels (left of the stem):\\
	\begin{tabular}{c|ccccccc}
		\lsptoprule 
		\diag{.1em}{.5cm}{\footnotesize{V1}}{\footnotesize{V2}} & i & ɪ & e & a & o & ʊ & u \\ 
		\cline{2-8}
		i & ii & yɪ & ye & ya & yo & yʊ & yu \\ 
		ɪ & ii & yɪ & ye & ya & yo & yʊ & yu \\ 
		%e & /  & /  & /  & /  & /  & /  & /  \\ 
		a & ii & ɪɪ & ee & aa & oo & ʊʊ & uu \\ 
		%o & /  & /  & /  & /  & /  & /  & / \\ 
		ʊ & wi & wɪ & we & wa & oo & ʊʊ & uu \\ 
		u & wi & wɪ & we & wa & oo & uu & uu \\ 
		\lspbottomrule
	\end{tabular}
\end{exe}
%\end{table}
\noindent Noun class 9 prefix \textit{jɪ}- is exceptional. Its vowel assimilates to any following vowel (one may alternatively analyse this as vowel deletion with subsequent compensatory lengthening). (\ref{exConcordNCL9beforeVowel}) illustrates this for the subject prefix.
\begin{exe}
	\ex \label{exConcordNCL9beforeVowel}
	\begin{tabbing}
		\textit{jʊʊmile}x\=(\degree jɪ-ʊm-ile)x\=\kill
		\textit{jiisile}\>(\degree jɪ-is-ile)\>`it (class 9) has come'\\
		\textit{jɪɪmile}\>(\degree jɪ-ɪm-ile)\>`it (class 9) has stopped'\\
		\textit{jeegile}\>(\degree jɪ-eg-ile)\>`it (class 9) has taken'\\
		\textit{jaagile}\>(\degree jɪ-ag-ile)\>`it (class 9) has found'\\
		\textit{joogile}\>(\degree jɪ-og-ile)\>`it (class 9) has bathed'\\
		\textit{jʊʊmile}\>(\degree jɪ-ʊm-ile)\>`it (class 9) has dried'
	\end{tabbing}
\end{exe}

Juxtaposition of heteromorphemic vowels to the right of the root (suffixation and infixation) occurs productively only in a limited number of cases, one of which is derivational suffixes attached to monosyllabic verbs. This yields slightly ideosyncratic results; see \sectref{VHHMonosyllabicVerbs}. The other two are the suffixing of inflectional affixes to monosyllabic verbs and a process of infixing referred to as \textit{imbrication} (\sectref{Imbrication}).\is{imbrication} Unlike with prefixes, the low vowel /a/ followed by a front vowel yields /ee/.
\begin{exe}
	\ex
	\begin{tabbing}
		xxxxx\=xxxx\=xxxxxx\=xxxxxxxxx\=xxxxxxxxxxxxxx\=xxxxxxx\kill
		i + i\>$\rightarrow$\>ii\>\textit{inamiike}\>(\degree inami<i>k-e)\>`bend (tr).\textsc{pfv}'\\
		ɪ + i\>$\rightarrow$\>ii\>\textit{pɪliike}\>(\degree pɪlɪ<i>k-e)\>`hear; feel.\textsc{pfv}'\\
		e + i\>$\rightarrow$\>ii\>\textit{kemiile}\>(\degree keme<i>l-e)\>`bark at.\textsc{pfv}'
	\end{tabbing}
	\ex\begin{tabbing}
		xxxxx\=xxxx\=xxxxxx\=xxxxxxxxx\=xxxxxxxxxxxxxx\=xxxxxxx\kill
		a + i\>$\rightarrow$\>ee\>\textit{peegwa}\>(\degree pa-igw-a)\>`be given'\\
		\>\>\>\textit{egeeme}\>(\degree ega<i>m-e)\>`lean.\textsc{pfv}'
		\\a + ɪ\>$\rightarrow$\>ee\>\textit{peela}\>(\degree pa-ɪl-a)\>`give away'
	\end{tabbing}
	\ex
	\begin{tabbing}
		xxxxx\=xxxx\=xxxxxx\=xxxxxxxxx\=xxxxxxxxxxxxxx\=xxxxxxx\kill
		o + e\>$\rightarrow$\>we\>\textit{bwene}\>(\degree bo<e>n-e)\>`see.\textsc{pfv}'\\
		o + i\>$\rightarrow$\>wi\>\textit{kosomwile}\>(\degree kosomo<i>l-e)\> \lq cough.\textsc{pfv}'\\
		ʊ + i\>$\rightarrow$\>wi\>\textit{alwike}\>(\degree alʊ<i>k-e)\>`stand up.\textsc{pfv}'\\
		u + i\>$\rightarrow$\>wi\>\textit{afwile}\>(\degree afu<i>l-e)\>`crawl.\textsc{pfv}'
	\end{tabbing}
\end{exe}
\is{vowels!hiatus solution|)}

\subsection{Consonants}
\is{consonants|(}
\subsubsection{Consonant inventory}
\tabref{TableConsonantInventory} shows the 16 phonemic consonants of Nyakyusa as they are spelt in this study. Where the phonetic value differs from the graphic representation the latter is given in angle brackets.
\begin{table}[H] % [H] sorgt dafür, dass genau hier und nicht verrutscht
	
	
	\begin{tabular}{|l|c|c|c|c|c|c|}
		\hline & 
		\footnotesize{Bilabial}&
		\footnotesize{Lab. dent}&
		\footnotesize{Alveolar}&
		\footnotesize{Palatal}&
		\footnotesize{Velar}&
		\footnotesize{Glottal}\\	
		%jeweils die, die sich berühren weg, d.h. bilabial c| lab. dent c| etc
		\hline Plosive &  				% Plosive
		p  &							% Bilabial
		&							% Labiodental
		t  &							% Alveolar
		ɟ <j> &					% Palatal
		k &							% Velar
		\\							% Glottal
		\hline Nasal & 					% Nasal
		m &							% Bilabial
		&							% Labiodental
		n &							% Alveolar
		ɲ <ny> &						% Palatal
		ŋ <ng'> &  					% Velar
		\\							% Glottal
		\hline Fricative & 				% Fricative
		& 							% Bilabial
		f &							% Labiodental
		s &							% Alveolar
		&							% Palatal
		&							% Velar
		h \\						% Glottal
		\hline Approx. &				% Approx.
		β̞ <b>   &     			% Bilabial
		&						% Labiodental
		&						% Alveolar
		&						% Palatal
		ɰ <g> &				% Velar
		\\						% Glottal
		\hline Lat. Appr. & 			% Lat. Approx
		&						% Bilabial
		&						% Labiodental
		l &			% Alveolar
		& 						% Palatal
		&						% Velar
		\\						% Glottal
		\hline Glide & 				% Glides
		&						% Bilabial
		&						% Labiodental
		&						% Alveolar
		(j) <y>  &				% Palatal
		(w) &					% Velar
		\\						% Glottal
		\hline
	\end{tabular}	
	\caption{Nyakyusa consonant inventory}
\label{TableConsonantInventory}	
\end{table}

The glottal fricative is very rare. In most cases it can be regularly traced back to loans from \ili{Kinga} and \ili{Kisi} \citep[218]{LabroussiC1998}. \citet{NurseD1979} even considers it to be so marginal as to make its phonemic status dubious. The velar nasal is frequent phonetically, but rare in its occurrence underlyingly. The approximants are realized as plosives when following a nasal and as approximants elsewhere. Lateral /l/ is optionally realized as [ɾ$\sim$ɹ] after a front vowel in unstressed syllables, unless it is followed by a glide. The palatal plosive is typically realized as an affricate [d͜ʒ] when following a nasal. Voiceless plosives tend to be aspirated. The fricative /f/ in native material can regularly be traced back to a diachronic or synchronic process of spirantization;\is{spirantization} see \citet{BoestonK2008} on Bantu spirantization. Morphophonemic processes can, however, obscure this relationship, detaching the fricative from the triggering vowel (see \sectref{ReciprocalAndCausative}, \ref{ApplicativizedCausatives}).\footnote{Note also the verb stem \textit{fifa} `hide (intr.)' (variant \textit{fisa}) < PB *\textit{píc}, a clear case of historical assimilation.} Nasals preceding another consonant are generally homorganic and will be written as <m> before a labial and as <n> elsewhere. Consonants in loans from Swahili that are not adapted to the phoneme inventory of Nyakyusa will be written with their respective phonetic symbols.

\subsubsection{Glides}\label{glides}
Glides can be synchronically or diachronically traced back to the desyllabification of a vowel,\is{vowels!hiatus solution} although the quality of the underlying vowel is not synchronically discernible in all cases. The palatalized alveolar nasal is clearly distinct from the palatal nasal. In monitored speech it is realized as [ni̯]. The palatalized alveolar nasal will be written as <ni> throughout this study in order to distinguish it from the palatal nasal /ɲ/, represented by the digraph <ny>.  Glides normally do not appear before rounded back vowels, although some lexicalized exceptions occur. Vowels following glides are slightly lengthened (\sectref{VowelLength}).

\subsubsection{Prenasalized plosives}\label{PrenasalizedPlosives}
Prenasalization in Nyakyusa is limited to plosives (see \tabref{tabPrenasalizedConsonants}).
\label{NC-Clusters}
\begin{table}[H] % [H] sorgt dafür, dass genau hier und nicht verrutscht
	\centering
	\begin{tabular}{cccc}
		\lsptoprule
		{\footnotesize{Bilabial}}&		% Bilabial
		{\footnotesize{Alveolar}} & 		% Alveolar
		{\footnotesize{Palatal}} & 		% Palatal
		{\footnotesize{Velar}} 	 \\		% Velar
		\midrule
		ᵐb <mb> & ⁿd <nd> & ⁿd͜ʒ <nj> & {\ᵑ}g <ng> \\
		\lspbottomrule
		
		%			\hline
	\end{tabular}	
	\caption{Prenasalized consonants}	
	\label{tabPrenasalizedConsonants} 
	%Fussnote bzgl. [y] statt [j]
\end{table}
Prenasalized plosives are always voiced and homorganic.\footnote{This voicing rule distinguishes Nyakyusa from the \ili{Ngonde} variety described by \citet[278]{LabroussiC1998}, although it does apply in the variety described by \citet{KishindoP1999}.} Voiceless plosives turn into their voiced counterparts when preceded by a non-syllabic nasal, and the approximants /β̞, l, ɰ/ into their plosive counterparts. (\ref{exnyclasses910}) shows that preceding vowels the prefix of noun class 9/10 surfaces as \textit{ny}- (note that any vowel preceding or following this prefix surfaces as long; see \sectref{NominalMorphology}) and (\ref{exNCclasses910}) illustrates how this prefix induces prenasalization.  (\ref{exNC1sg}) illustrates prenasalization with the subject prefix of the first person singular.\footnote{Before certain object prefixes, the \textsc{1sg} subject prefix turns into a syllabic nasal; see \sectref{SubjectConcordsParticipants}.}
\begin{exe}
	\ex \label{exnyclasses910}
	\begin{tabbing}
		xxxxxxxxx\=xxxxxxxxxxxx\=xxxxxxxxxxxxxx\=xxxxxxx\kill 
		\textit{ɪɪnyiifi}\>(\degree ɪ-ny-ifi)\>`chameleons'
		\\\textit{ɪɪnyɪmbo}\>(\degree ɪ-ny-ɪmbo)\>`songs'
		\\\textit{ɪɪnyendo}\>(\degree ɪ-ny-endo)\>`journeys'
		\\\textit{ɪɪnyaala}\>(\degree ɪ-ny-ala)\>`grindstones'
		\\\textit{ɪɪnyoobe}\>(\degree ɪ-ny-obe)\>`fingers'
		\\\textit{ɪɪnyʊʊbo}\>(\degree ɪ-ny-ʊbo)\>`big knives'
	\end{tabbing}
	              \newpage 
	\ex \label{exNCclasses910}
	\begin{tabbing}
		xxxxxxxxx\=xxxxxxxxxxxx\=xxxxxxxxxxxxxx\=xxxxxxx\kill 
		\textit{ɪmbepo}\>(\degree ɪ-ny-pepo)\>`cold; air(s); spirit(s)'
		\\\textit{ɪndʊmi}\>(\degree ɪ-ny-tʊmi)\>`message(s), news'
		\\\textit{ɪnjʊni}\>(\degree ɪ-ny-jʊni)\>`bird(s)'
		\\\textit{ɪnguuto}\>(\degree ɪ-ny-kuuto)\>`cry / cries'
		\\\textit{ɪmbatɪko}\>(\degree ɪ-ny-batɪko)\>`procedure(s)'
		\\\textit{ɪndagɪlo}\>(\degree ɪ-ny-lagɪlo)\>`law(s); commission(s)'
		\\\textit{ɪngolo}\>(\degree ɪ-ny-golo)\>`louse(s)'
	\end{tabbing}
	\ex\label{exNC1sg}
	\begin{tabbing}
		xxxxxxxxx\=xxxxxxxxxxxx\=xxxxxxxxx\=xxxxxxxxxxxxxx\=xxxxxxx\kill 
		\textit{mbinyile}\>(\degree n-piny-ile)\>`I have bound' 
		\\\textit{ndaagile}\>(\degree n-taag-ile)\>`I have thrown'
		\\\textit{njaatile}\>(\degree n-jaat-ile)\>`I have taken a walk' 
		\\\textit{ngeetile}\>(\degree n-keet-ile)\>`I have watched'
		\\\textit{mbalile}\>(\degree n-bal-ile)\>`I have counted'
		\\\textit{ndɪlile}\>(\degree n-lɪl-ile)\>`I have cried'
		\\\textit{ngelile}\>(\degree n-gel-ile)\>`I have tested'
	\end{tabbing}
\end{exe}

Prenasalized consonants are not considered phonemic for a number of reasons. First, when they occur across morpheme boundaries, as in the above examples, they can clearly be analysed as the result of a nasal segment followed by a plosive or approximant. Second, the nasals involved are also found as single phonemes and the voiced plosives are regular allophones of /β̞, l, ɰ/ in a post-nasal context, as can also be seen with the syllabic nasals (\sectref{SyllabicNasals}). Further, their stem-internal distribution is very limited, the majority of cases being the second consonant of the stem. Last, they are rarely found in affixes. All of this speaks in favour of assuming syllable structures of the type /NC\ldots / rather than an additional set of phonemes. Note that prenasalization triggers lengthening of a preceding vowel,\is{vowels!length} as discussed in \sectref{VowelLength}.

\subsubsection{Syllabic nasals}\label{SyllabicNasals}
In the great majority of cases, syllabic nasals are the result of a series of morphophonemic processes affecting noun class affixes. The most frequently applying of these processes is deletion of the underlying high back vowel of the prefixes of noun classes 1, 3, and 18 in most pre-consonantal environments and subsequent syllabification of the nasal segment. Other sources include the first person singular subject prefix in determined phonemic and morphological contexts (see \sectref{SubjectConcordsParticipants}).

Syllabic nasals hardly ever appear morpheme-internally. Exceptions include \textit{mma} [ˈm̩.ma] `no' and related forms such as \textit{somma} [sɔ.ˈm̩.ma] `don't!', as well as \textit{nnoono} [n̩.ˈnɔː.nɔ] `so much' and \textit{nsyɪsyɪ} [n̩.ˈsʸɪˑ.sʸɪ] `skunk' (pl. \textit{bansyɪsyɪ}). In contrast to prenasalized plosives, syllabic nasals do not trigger voicing. Thus, in the graphic representation, the syllabicity of nasals is ambiguous only in sequences involving a voiced plosive, and will be marked <m̩, n̩> in this context. Further, syllabic nasals induce shortening of preceding long or lengthened vowels (see \sectref{VowelLength}).\is{vowels!length} Phonetically they are realized with a longer duration than non-syllabic nasals. As with prenasalized plosives, syllabic nasals as a general rule are homorganic to the following segment, and approximant phonemes are hardened to plosives. A syllabic nasal preceding /h/ is realized as velar. 
\begin{exe}
	\ex\begin{tabular}[t]{lll}
		\textit{ʊm̩belo}&(\degree ʊ-mu-belo)&`wind'\\
		\textit{ʊn̩dʊme}&(\degree ʊ-mu-lʊme)&`husband'\\
		\textit{ʊn̩gʊnda}&(\degree ʊ-mu-gʊnda)&`field, farm'\\
		\textit{ʊnhɪɪji}&(\degree ʊ-mu-hɪɪj-i)&`thief'
	\end{tabular}
\end{exe}
\is{consonants|)}
\subsection{Suprasegmentals}\label{Stress} 
\is{stress|(}Nyakyusa is not a tonal language, unlike the majority of Bantu languages (\citealt{KisseberthCOddenD2003} among others). The loss of inherited tone in Nyakyusa has been noted even in early studies (e.g. \citealt{NurseD1979}; \citealt{GuthrieM1967}; \citealt{vanEssenOKaehler-MeyerE1969}). Instead, Nyakyusa features a regular penultimate accent. The lack of phonemic tone makes Nyakyusa quite different from its close relative \ili{Ndali} (see e.g. \citealt{NurseD1988}; \citealt{BotneR2008}), although this characteristic is apparently shared with \ili{Ngonde} (\citealt{KishindoP1999}; \citealt{LabroussiC1999})\is{tone}. Every word in Nyakyusa is assigned stress on the penultimate syllable. However, words are often grouped together as a prosodic unit, in which case the penultimate syllable of the rightmost word receives a more pronounced stress.\footnote{See \citet{BickmoreLClemensL2016} on \ili{Tooro} JE12 for a similar case.} Typical cases are nouns followed by a determiner:
\begin{exe}
	\ex
	\glll ʊˌmwan(a) ˈʊjʊ\\
	ʊ-mu-ana ʊ-jʊ\\
	\textsc{aug}-1-child \textsc{aug}-\textsc{prox.1}\\
	\glt `this child'
	\ex
	\glll iiˌgalɪ ˈlyangʊ\\
	ii-galɪ lɪ-angʊ\\
	5-car 5-\textsc{poss.1sg}\\
	\glt \lq my car'
\end{exe}

In the same fashion, a determiner introducing a relative clause (see \sectref{RelativeClausss}) forms a prosodic unit with the following verb:
\begin{exe}
	\ex \glll ʊn̩ˈdʊngʊ ˌʊgʊ gʊˈkwisa aˈtwale ɪɪˌheela \\
	ʊ-mu-lʊngʊ ʊ-gʊ gʊ-kʊ-is-a a-twal-e ɪɪ-heela\\
	\textsc{aug}-3-week \textsc{aug}-\textsc{prox.3} 3-\textsc{prs}-come-\textsc{fv} 1-carry-\textsc{subj} \textsc{aug}-money(9)\\
	\sn \glll ˈjangʊ\\
	jɪ-angʊ\\
	9-\textsc{poss.1sg}\\
	\glt `Next week [lit. the week that comes] he must bring my money.' [Monkey and Tortoise]
\end{exe}
\is{stress|)}
\section{Nouns and noun phrase}
\subsection{Noun classes}\label{NounClasses}
\is{noun classes|(}Bantu languages are well-known for their noun class systems; for an introduction see \citet[ch. 3]{MahoJ1999} and \citet{KatambaF2003}. Nyakyusa has a typical system of 18 noun classes, of which some are further differentiated into subclasses. Each noun class is characterized by a series of agreement prefixes. Agreement in the noun phrase occurs between the head and its modifiers, as (\ref{exAgreementNounPhrase}) illustrates. See \citet{LusekeloA2009a} on the linear structure of the Nyakyusa noun phrase.

\begin{exe}
	\ex\label{exAgreementNounPhrase}
	\gll a-\textbf{ba}-ana a-\textbf{ba}-lʊmyana a-\textbf{ba}-tupe \textbf{b}-angʊ \textbf{ba}-bɪlɪ \textbf{ba}-la\\
	\textsc{aug}-2-child \textsc{aug}-2-boy \textsc{aug}-2-fat 2-\textsc{poss.1sg} 2-two  2-\textsc{dist}\\
	\glt `those two fat sons of mine' [ET]
\end{exe}

As to agreement marking, two sets of prefixes can be distinguished within the noun phrase:

\begin{itemize}
	\item The \textit{nominal prefix} (\textsc{npx}) is used with nouns and adjectives and the numerals 2--5, as well as with the bound roots \textit{nandɪ} `little, few', \textit{ingi} `many, much', \textit{lɪnga} `how much, how many', and \textit{ki} `what kind of, which'.
	\item The \textit{pronominal prefix} (\textsc{ppx}) is used to form demonstratives (see \sectref{Demonstratives}), and also occurs with the associative and possessives as well as with the bound roots \textit{mo} `one, some', \textit{ngɪ} `other', \textit{ope} `also', \textit{ene} `self; owner' (except for class 1) and \textit{osa} `all'. It is also used as a secondary prefix (see \sectref{NominalMorphology}).
\end{itemize}

As the list indicates, there are a number of additional complexities concerning the choice and shape of these agreement prefixes, of which only the most frequent can be discussed here. In noun class 1, the pronominal prefix is \textit{gʊ}- with possessives, the associative and \textit{osa} `all' (yielding irregular \textit{gwesa}), and \textit{jʊ}- elsewhere. Further, the bound root \textit{ene} `self; owner' takes the nominal prefix in noun class 1, but the pronominal prefix in all other classes. Numerals in class 10 are marked by a prefix \textit{i}-.

Subclasses (1a, 2a, 9a, 10a) differ from their main class in the marking of the head noun -- in the case of 1a, 9a and 10a they lack an overt nominal prefix altogether. Dependent constituents, however, take the agreement forms of the respective main classes. Throughout this study, subclasses will not be explicitly marked in glossing. The agreement prefixes are further subject to a number of morphophonological alternations, some of which will be discussed in \sectref{NominalMorphology}. 

Nouns, adjectives and a number of other bound roots can also carry the augment, which is commonly referred to as the pre-prefix or initial vowel. This morpheme has the shape of a single vocalic segment /ɪ, a, ʊ/, whose place of articulation harmonizes with that of the respective noun class's underlying prefix vowel. See \sectref{NominalMorphology} for a short discussion of the distribution of the augment.

The semantics of the Nyakyusa noun classes are only semi-transparent and can best be described in terms of some common core meanings.

\tabref{TableSemanticsNCL} gives an overview of the various noun classes in Nyakyusa, their nominal agreement prefixes and frequent semantic elements of each class; cross-reference markers on the verb will be discussed in \sectref{SubjectConcords}, \ref{ObjectConcords}, \ref{LocativeEnclitics}. The noun classes are numbered according to the common Bleek-Meinhof system. The ascribed meanings are based on a synthesis of previous grammatical sketches\footnote{\citet{SchumannK1899},  \citet{MeinhofC1966}, \citet{EndemannC1914}, \citet{NurseD1979}, \citet{LusekeloA2007} and \citet{FelbergK1996}} and have been refined by the author through the inclusion of subclasses.


\begin{table}
\small
\begin{tabularx}{\textwidth}{lllll}
	\lsptoprule
	\footnotesize{Class} & \footnotesize{AUG} & \footnotesize{NPx} & \footnotesize{PPx} & \footnotesize{Semantics} \\ 
	\midrule
	
	1 & \textit{ʊ}- & \textit{mu}- & \textit{jʊ}- (\textit{gʊ}-) & human beings \\
	
	1a & (\textit{ʊ}-) & \textit{ø}- & & kinship terms; proper names
	\tabularnewline &  & &  & some living beings; some loans\\
	
	2 & \textit{a}- & \textit{ba}- & \textit{ba}- & regular plural of class 1 \\ 
	
	2a & (\textit{a})- & \textit{baa}- & & regular plural of class 1a\\
	
	3 & \textit{ʊ}- & \textit{mu}- & \textit{gʊ}- & plants, other non-animates
	\tabularnewline &  & &  &  natural phenomena \\ 
	
	4 & \textit{ɪ}- & \textit{mi}- & \textit{gɪ}- & regular plural of classes 3 and 14
	\tabularnewline &  & &  & regular plural of class 5 augmentatives\\ 
	
	5 & \textit{ɪ}- & \textit{li}- (\textit{ii}-) & \textit{lɪ}- & fruits, produce; body parts; miscellanea
	\tabularnewline &  & &  &  augmentatives \\ 
	
	6 & \textit{a}- & \textit{ma}- & \textit{ga}- & regular plural of class 5
	\tabularnewline &  & &  &  mass terms and liquids; paired objects \\
	
	7 & \textit{ɪ}- & \textit{kɪ}- & \textit{kɪ}- & characteristics, mannerisms, languages
	\tabularnewline &  & &  & instruments, utensils, tools
	\tabularnewline & & & & derogatives \\ 
	
	8 & \textit{ɪ}- & \textit{fi}- & \textit{fi-} & regular plural of class 7 \\ 
	
	9 & \textit{ɪ}- & \textit{ny}- & \textit{jɪ}- & animals; frequently used objects \tabularnewline &  & &  & some abstracts \\ 
	
	9a & \textit{ɪɪ}- & ø- &  & primarily loans \\ 
	
	10 & \textit{ɪ}- & \textit{ny}- & \textit{si}- & regular plural to class 9 
	\tabularnewline & & & &plural to concrete nouns of class 11 \\ 
	
	10a & \textit{ɪɪ}- & ø- & & regular plural to class 9a \\ 
	
	11 & \textit{ʊ}- & \textit{lʊ}- & \textit{lʊ}- & long, thin entities; some abstracts \tabularnewline & & &  & single instances of collectives
	\tabularnewline &  & &  & particular instances of class 9 stems \\ 
	
	12 & \textit{a}- & \textit{ka}- & \textit{ka}- & large quantities; some miscellanea
	\tabularnewline &  & &  & diminutives \\ 
	
	13 & \textit{ʊ}- & \textit{tʊ}- & \textit{tʊ}- & regular plural of class 12 \\ 
		
	14 & \textit{ʊ}- & \textit{bʊ}- & \textit{bʊ}- & qualities, characteristics, materials
	\tabularnewline &  & &  & abstracts; localities, countries
	\tabularnewline &  & &  &  some concretes \\
	
	15 & \textit{ʊ}- & \textit{kʊ}- & \textit{kʊ}- & infinitives \\ 
	
	16 & / & \textit{pa}- & \textit{pa}- & \isi{locative}: \lq at', `proximity' \\ 
	
	17 & / & \textit{kʊ}- & \textit{kʊ}- & locative: \lq general area', `far away' \\ 
	
	18 & / & \textit{mu}- & \textit{mu}- & locative: `in' \\ 
	\lspbottomrule 
\end{tabularx}
\caption[]{Nyakyusa noun classes} \label{TableSemanticsNCL}
\end{table}

As can be gathered from \tabref{TableSemanticsNCL}, Nyakyusa noun classes form a number of regular singular-plural pairings. \figref{FIGattestedsgplpairings} illustrates these pairings, which for the main classes represent one of the most frequent patterns across Bantu \citep[109]{KatambaF2003}. Examples for each of these pairings are given in (\ref{exNCLpairings}). See (\ref{exLocativeNouns}) below for examples of the locative noun classes and Chapter \ref{ChapterInfinitives} on the infinitive noun class 15. Note that when it comes to nominal agreement, the discourse participants (first and second person) fall within noun classes 1 (singular) and 2 (plural).

\begin{figure}[h]
	\begin{center}
		\begin{displaymath}
		\xymatrix@R=2.75pt@C=2in{    %@R= spaces rows from each other
			\textbf{Sg.}            &    \textbf{Pl.}\\
			1    \ar[r] 		        &    2\\
			1\textrm{a}   \ar[r]                &    2\textrm{a}\\
			3    \ar[r]                 &    4\\
			5    \ar[r]\ar[ur]          &    6 \\
			7    \ar[r]                 &    8\\
			9    \ar[r] 		        &    10\\
			9\textrm{a}	 \ar[r]         &    10\textrm{a}\\
			11    \ar[uur]	 		     &    \\
			12    \ar[r]                &    13\\
			14    \ar[uuuuuuur]            &    \\
		}
		\end{displaymath}
		\caption{Attested noun class pairings}\label{FIGattestedsgplpairings}
	\end{center}
\end{figure}

\newpage 
\begin{exe}
\ex \label{exNCLpairings}
\begin{xlist}
\ex Classes 1/2:
\begin{tabbing}
\textit{ʊlʊnywili}x\=(\degree ɪ-ny-nywili)x\=\kill %nonsense, for tabbing
\textit{ʊmwana} \> (\degree ʊ-mu-ana) \> \lq child'\\
\textit{abaana} \> (\degree a-ba-ana) \> \lq children'
\end{tabbing}
\ex Classes 1a/2a:
\begin{tabbing}
\textit{ʊlʊnywili}x\=(\degree ɪ-ny-nywili)x\=\kill %nonsense, for tabbing
\textit{taata} \> (\degree ø-ø-taata) \> \lq my father'\\
\textit{baataata} \> (\degree ø-baa-taata) \> \lq my fathers'
\end{tabbing}
\ex Classes 3/4:
\begin{tabbing}
\textit{ʊlʊnywili}x\=(\degree ɪ-ny-nywili)x\=\kill %nonsense, for tabbing
\textit{ʊmpiki} \> (\degree ʊ-mu-piki) \> \lq tree'\\
\textit{ɪmipiki} \> (\degree ɪ-mi-piki) \> \lq trees'
\end{tabbing}
\ex Classes 5/6:
\begin{tabbing}
\textit{ʊlʊnywili}x\=(\degree ɪ-ny-nywili)x\=\kill %nonsense, for tabbing
\textit{iitooki} \> (\degree ii-tooki) \> \lq type of banana'\\
\textit{amatooki} \> (\degree a-ma-tooki) \> \lq bananas'
\end{tabbing}
\ex Classes 5/4 (augmentatives):
\begin{tabbing}
\textit{ʊlʊnywili}x\=(\degree ɪ-ny-nywili)x\=\kill %nonsense, for tabbing
\textit{iibwa} \> (\degree ii-bwa) \> \lq big/bad dog'\\
\textit{ɪmibwa} \> (\degree ɪ-mi-bwa) \> \lq big/bad dogs'
\end{tabbing}
\ex Classes 7/8:
\begin{tabbing}
\textit{ʊlʊnywili}x\=(\degree ɪ-ny-nywili)x\=\kill %nonsense, for tabbing
\textit{ɪkɪkombe} \> (\degree ɪ-kɪ-kombe) \> \lq cup'\\
\textit{ɪfikombe} \> (\degree ɪ-fi-kombe) \> \lq cups'
\end{tabbing}
\ex Classes 9/10:
\begin{tabbing}
\textit{ʊlʊnywili}x\=(\degree ɪ-ny-nywili)x\=\kill %nonsense, for tabbing
\textit{ɪmbwa} \> (\degree ɪ-ny-bwa) \> \lq dog'\\
\textit{ɪmbwa} \> (\degree ɪ-ny-bwa) \> \lq dogs'
\end{tabbing}
\ex Classes 9a/10a:
\begin{tabbing}
\textit{ʊlʊnywili}x\=(\degree ɪ-ny-nywili)x\=\kill %nonsense, for tabbing
\textit{ɪɪpʊsi} \> (\degree ɪɪ-ø-pʊsi) \> \lq cat' (<EN)\\
\textit{ɪɪpʊsi} \> (\degree ɪɪ-ø-pʊsi) \> \lq cats'
\end{tabbing}
\ex Classes 11/10:
\begin{tabbing}
\textit{ʊlʊnywili}x\=(\degree ɪ-ny-nywili)x\=\kill %nonsense, for tabbing
\textit{ʊlʊnywili} \> (\degree ʊ-lʊ-nywili) \> \lq hair (sg.)'\\
\textit{ɪɪnywili} \> (\degree ɪ-ny-nywili) \> \lq hair (pl.)'
\end{tabbing}
\ex Classes 12/13:
\begin{tabbing}
\textit{ʊlʊnywili}x\=(\degree ɪ-ny-nywili)x\=\kill %nonsense, for tabbing
\textit{akapango} \> (\degree a-ka-pango) \> \lq story'\\
\textit{ʊtʊpango} \> (\degree ʊ-tʊ-pango) \> \lq stories'
\end{tabbing}
\ex Classes 12/13 (diminutives):
\begin{tabbing}
\textit{ʊlʊnywili}x\=(\degree ɪ-ny-nywili)x\=\kill %nonsense, for tabbing
\textit{akabwa} \> (\degree a-ka-bwa) \> \lq little dog'\\
\textit{ʊtʊbwa} \> (\degree ʊ-tʊ-bwa) \> \lq little dogs'
\end{tabbing}
\ex Classes 14/4:
\begin{tabbing}
\textit{ʊlʊnywili}x\=(\degree ɪ-ny-nywili)x\=\kill %nonsense, for tabbing
\textit{ʊbooga} \> (\degree ʊ-bʊ-oga) \> \lq mushroom'\\
\textit{ɪmyoga} \> (\degree ɪ-mi-oga) \> \lq mushrooms'
\end{tabbing}
\end{xlist}	
\end{exe}

\subsection{Nominal morphology}\label{NominalMorphology}
The linear structure of a canonical nominal in Nyakyusa can be schematized as in \figref{FigureStructureNominals}.
\begin{figure}[H] % [H] sorgt dafür, dass genau hier und nicht verrutscht
	\centering
	\begin{tabularx}{8cm}{ccc}
		\hline 
		(Augment) & \multirow{3}{*}{Nominal prefix} & \multirow{3}{*}{Stem} \\ 
		(Locative prefix)  &  & \\(Pronominal prefix) & &\\ 
		\hline 
	\end{tabularx}
	\caption{Structure of Nyakyusa nominals}
	\label{FigureStructureNominals} 
\end{figure} 

The grammatical functions of the augment in Bantu are complex and need further research; see \citeauthor{HymanLKatambaF1991} (\citeyear{HymanLKatambaF1991}; \citeyear{HymanLKatambaF1993}) and also \citet{VanDerWalJNamyaloS2015} for insightful discussions of a number of pragmatic and syntactic functions of the augment in \ili{Ganda} JE15. In Nyakyusa use of the augment is optional, although it is excluded in a number of contexts: following the associative -\textit{a}, in predicative and vocative use, following \textit{ngatɪ} `as, like', \textit{kʊkʊtɪ} `every' and (\textit{ɪ})\textit{kɪsita}/(\textit{ʊ})\textit{kʊsita} \lq without', and when the head noun is modified by \textit{ki} `which, what kind of'. Nouns carrying a pronominal prefix instead of an augment, express an emphatic notion translatable along the lines of `just X; \mbox{the very X}':

\begin{exe}
	\ex\begin{tabular}[t]{ll}
		\textit{lʊ}-\textit{lw}-\textit{ala} & `the very grindstone (class 11)'\\
		\textit{gʊ}-\textit{n}-\textit{tʊ} & `the very head (class 3)'\\
	\end{tabular}
\end{exe}

Inherently \isi{locative} nouns are rare. Locatives\is{locative} are commonly formed by additive marking, that is by prefixing a locative noun class prefix to a noun already marked for a noun class of its own:
\begin{exe}
	\ex \label{exLocativeNouns}
	\begin{tabular}[t]{ll}
		\textit{pa}-\textit{lw}-\textit{ɪsi} & `at (class 16) a/the river'\\
		\textit{kʊ}-\textit{lw}-\textit{ɪsi} & `at (class 17) a/the river'\\
		\textit{mu}-\textit{lw}-\textit{ɪsi} & `in (class 18) a/the river'\\
	\end{tabular}
\end{exe}

In a manner analogous to the prefixing of pronominal prefixes discussed above, \isi{locative} nouns may carry more than one prefix of the same \isi{locative} class in addition to the noun's basic noun class prefix. This twofold or even threefold locative marking seems to give emphasis to the specific \isi{locative} semantics.
\begin{exe}
	\ex
	\begin{tabular}[t]{ll}
		\textit{pa}-\textit{pa}-\textit{kɪ}-\textit{lo} & `\textit{on that very} (class 16) night'
		\\\textit{mu}-\textit{n}-\textit{k}-\textit{iina} &  `\textit{in} (class 18) the pit'
		\\\textit{mu}-\textit{mu}-\textit{n̩}-\textit{dw}-\textit{ɪsi} & \lq \textit{in} (class 18) the river'
	\end{tabular}
\end{exe}

The nominal prefixes (\sectref{NounClasses}) are subject to a number of morphophonological alternations, of which only the most general ones can be discussed here. The nominal prefix \mbox{\textit{mu}-} of noun classes 1, 3 and 18 is commonly realized as a syllabic nasal when preceding consonants other than the prenasalized plosives; see \sectref{SyllabicNasals} for some examples. The nominal prefix \textit{ny}- of classes 9/10 does not surface when preceding a stem-initial nasal or voiceless fricative, unless the resultant word would be monosyllabic, in which case it surfaces as a syllabic nasal (\ref{exNPXnyDeletionOrNot}). Any prefix preceding an underlying \textit{ny}- (the augment, a locative prefix or a pronominal prefix) is realized with a long vowel. The same holds for any following stem-initial vowel (\ref{exNPXnyVowelsLong}). Similarly, augments and other prefixes on subclass 9a/10a nouns always feature a long vowel (\ref{exAugment9a10along}).

\begin{exe}
	\ex \label{exNPXnyDeletionOrNot}
	\begin{tabbing}
		\textit{kʊʊnyuma}x\=(\degree kʊ-ny-nyuma)x\=\lq affair(s); case(s)'xx\=cf.\textit{akagombe}x\=\kill%Unsinnszeile für Tabulatoren
		\textit{ɪɪnongwa}\>(\degree ɪ-ny-nongwa)\>\lq affair(s); case(s)'\\
		\textit{ɪɪswi}\>(\degree ɪ-ny-swi)\>`fish(es)'\\
		\textit{nswi}\>(\degree ny-swi)\>`it is (a/the) fish(es)'
	\end{tabbing}
	\ex \label{exNPXnyVowelsLong}
	\begin{tabbing}
		\textit{kʊʊnyuma}x\=(\degree kʊ-ny-nyuma)x\=`affair(s); case(s)'xx\=cf.\textit{akagombe}x\=\kill%Unsinnszeile für Tabulatoren
		\textit{ɪɪnyuma}\>(\degree ɪ-ny-nyuma)\>`back'\\
		\textit{jɪɪnyuma}\>(\degree jɪ-ny-nyuma)\>`the very back'\\
		\textit{kʊʊnyuma}\>(\degree kʊ-ny-nyuma)\>`behind, after'\\
		\textit{ɪɪnyaala}\>(\degree ɪ-ny-ala)\>`grindstones'
	\end{tabbing}
	\ex\label{exAugment9a10along}
	\begin{tabbing}
		\textit{kʊʊnyuma}x\=(\degree kʊ-ny-nyuma)x\=`affair(s); case(s)'xx\=cf.\textit{akagombe}x\=\kill
		\textit{ɪɪlefani}\>(\degree ɪ-lefani)\>`spoon'\\
		\textit{ɪɪpʊsi}\>(\degree ɪ-pʊsi)\>\lq cat'
	\end{tabbing}
\end{exe}

Further, there are a few class 9/10 stems where the prefix and the stem-initial consonant fuse due to a fossilized morphophonemic process known as Meinhof's law (\ref{exMeinhofsLaw}). In noun class 5, with consonant-initial stems, most commonly a prefix \textit{ii}- is used, with no distinction concerning the presence or absence of an augment (\ref{exPrefixNCL5}).
\begin{exe}
	\ex\label{exMeinhofsLaw}
	\begin{tabbing}
		\textit{ɪɪnyumba}x\=(\degree ɪ-ny-jumba)x\=`leopard'xx\=cf.\textit{akagombe}x\=\kill%Unsinnszeile für Tabulatoren
		\textit{ɪɪng'ombe}\>(\degree ɪ-ny-gombe)\>`cow'\>cf. \textit{akagombe}\>`little cow (class 12)'
		\\\textit{ɪɪnyumba}\>(\degree ɪ-ny-jumba)\>`house'\>cf. \textit{akajumba}\>`little house (class 12)'
	\end{tabbing}
	\ex\label{exPrefixNCL5}
	\begin{tabbing}
		\textit{ɪɪnyumba}x\=(\degree ɪ-ny-jumba)x\=`leopard'xx\=cf. .\textit{akagombe}x\=\kill%Unsinnszeile für Tabulatoren
		\textit{ii}-\textit{bole}\>`leopard'\\
		\textit{ii}-\textit{lʊʊka}\>`shop'\\
		\textit{ii}-\textit{syʊ}\>`word'
	\end{tabbing}
\end{exe}

\subsection{Demonstratives and pronominals}\label{Demonstratives} 


Three basic demonstratives can be distinguished. First, a proximal of the shape \textsc{aug}-\textsc{ppx}.\footnote{Sometimes (but not consistently) a Swahili-influenced variant /hVCV/ is heard: e.g. \textit{hɪkɪ} `this (7)'.} This demonstrative is always used with the augment. Second, a demonstrative with a referential function (\textsc{aug}-\textsc{ppx}-\textit{o}). Used without the augment, this further serves as an emphatic copulative. Following common Bantuist terminology, throughout this study these demonstratives will also be called substitutives. Third, there is a distal demonstrative (\textsc{ppx}-\textit{la}). \tabref{TableDemonstratives} illustrate these forms. 

\begin{table}
	\begin{center}
			\begin{tabular}{cccccccc}
			\lsptoprule 
			\footnotesize{Class} & \footnotesize{Proximal} & \footnotesize{Referential} & \footnotesize{Distal} & \footnotesize{Class} & \footnotesize{Proximal} & \footnotesize{Referential} & \footnotesize{Distal} \\ 
			\cmidrule(r){1-4}\cmidrule(l){5-8}
			1 & \textit{ʊjʊ} & \textit{ʊ-jo} & \textit{jʊla} & 11 & \textit{ʊlʊ} & \textit{ʊ-lo} & \textit{lʊla}\\ 
			2 & \textit{aba} & \textit{a-bo} & \textit{bala} & 12 & \textit{aka} & \textit{a-ko} & \textit{kala} \\
			3 & \textit{ʊgʊ} & \textit{ʊ-go} & \textit{gʊla} & 13 & \textit{ʊtʊ} & \textit{ʊ-to} & \textit{tʊla}\\
			4 & \textit{ɪgɪ} & \textit{ɪ-gyo} & \textit{gɪla} & 14 & \textit{ʊbʊ} & \textit{ʊ-bo} & \textit{bʊla}\\
			5 & \textit{ɪlɪ} & \textit{ɪ-lyo} & \textit{lɪla} & 15 & \textit{ʊkʊ} & \textit{ʊ-ko} & \textit{kʊla}\\
			6 & \textit{aga} & \textit{a-go} & \textit{gala} & 16 & \textit{apa} & \textit{a-po} & \textit{pala}\\
			7 & \textit{ɪkɪ} & \textit{ɪ-kyo} & \textit{kɪla} & 17 & \textit{kʊno} & \textit{ʊ-ko} & \textit{kʊla} \\
			8 & \textit{ɪfi} & \textit{ɪ-fyo} & \textit{fila} & 18 & \textit{muno} & \textit{ʊ-mo} & \textit{mula}\\
			9 & \textit{ɪjɪ} & \textit{ɪ-jo} & \textit{jɪla} \\
			10 & \textit{ɪsi} & \textit{ɪ-syo} & \textit{sila} &\\
			\lspbottomrule 
		\end{tabular}
		\caption{Noun class demonstratives}
		\label{TableDemonstratives}
	\end{center}
\end{table}




Numerous emphatic forms can be constructed through various patterns of reduplication,\is{reduplication} for instance 
\textit{kɪkɪɪkɪ} `this very one (class 7)', \textit{kɪɪkyo} `the very one (class 7) (e.g. already mentioned)' or \textit{kɪlakɪla} `that very one (class 7)'. Locative classes 17 and 18 (but not 16) differ in that their proximal demonstrative has the shape \textsc{ppx}-\textit{no}. Parallel forms also exist for class 14 (\textit{bʊno} `this way, thus, so') and class 5 (\textit{lɪno}, \textit{lɪlɪno} `now; today').



A number of demonstratives have acquired special meanings and functions. The class 14 substitutive \textit{bo} serves a number of circumstancial functions, such as introducing adverbials of comparison. It also has an aspectual function of establishing or reintroducing a temporal anchor (\ref{exboreference}) and introducing temporal adverbial clauses\is{subordinate clauses!temporal clause} (\ref{exbo}).

\begin{exe}
	\ex \label{exboreference}
	\gll bo a-a-bomb-aga fi-ki? \\
	\textsc{ref}.14 1-\textsc{pst}-do-\textsc{ipfv} 8-what\\
	\glt `What was he doing then?' [ET]
    \ex \label{exbo} \gll bo fi-kw-and-a ʊ-kʊ-bɪfw-a, ɪ-n-gambɪlɪ si-lɪnkw-and-a ʊ-kʊ-ly-a ɪ-fi-lombe m-mi-gunda gy-a ba-ndʊ ba-la \\ 
    \textsc{ref}.14 8-\textsc{prs}-begin-\textsc{fv} \textsc{aug}-15-ripen-\textsc{fv}, \textsc{aug}-10-monkey 10-\textsc{narr}-begin-\textsc{fv} \textsc{aug}-15-eat-\textsc{fv}
 \textsc{aug}-8-maize 18-4-farm 4-\textsc{assoc} 2-person 2-\textsc{dist} \\
	\glt `When it began to ripen, monkeys started to eat the maize in those people's field.' [Thieving monkeys]
\end{exe}

Similarly, the \isi{locative} determiners are used as temporals. Further cases of specialized meanings include class 11 proximal \textit{ʊlʊ} `in this way; now' and class 18 reduplicated substitutives \textit{muumo, momuumo} \lq right, all right; accordingly; complete'.

The pronouns for the \isi{discourse participants} (substitutives) are given in \tabref{TablePersonalPronouns}. In contrast to the substitutives of the noun classes, they have the shape \textsc{aug}-C(G)-\textit{e}. Note the change of the consonant in the first person plural when preceded by a vowel.

\begin{table} 
	\begin{center}
			\begin{tabularx}{4cm}{cc}
			\lsptoprule 
			\footnotesize{Participant} & \footnotesize{Pronoun} \\ 
			\midrule
			\textsc{1sg} & \textit{ʊ-ne} \\ 
			\textsc{2sg} & \textit{ʊ-gwe} \\
			\textsc{1pl} & \textit{twe}, \textit{ʊ-swe} \\
			\textsc{2pl} & \textit{ʊ-mwe} \\
			\lspbottomrule 
		\end{tabularx} 
	\caption{Participant pronouns}\label{TablePersonalPronouns}
	\end{center}
\end{table}

\newpage
\noindent Emphatic personal pronouns are formed with a reduplicated class 1 pronominal prefix. This holds for the singular as well as for the plural: \textit{jʊ}$\sim$\textit{jʊʊ}-\textit{ne} `just me', \textit{jʊjʊʊgwe} `just you', \textit{jʊjʊʊswe} `just us', \textit{jʊjʊʊmwe} `just you (pl.)'. \is{noun classes|)}

\section{Basic syntax}
\is{syntax|(}The basic word order in Nyakyusa is Subject--(Auxiliary)--Verb--1\textsuperscript{ary} Object--2\textsuperscript{ary} Object (--Adjuncts).
\begin{exe}
	\ex \gll a-lɪ pa-kʊ-ba-p-a a-ba-ana ɪ-fi-ndʊ pa-ka-aja\\
	1-\textsc{cop} 16-15-2-give-\textsc{fv} \textsc{aug}-2-child \textsc{aug}-8-food 16-12-homestead\\
	\glt `She is giving the children food at home.' [ET]
\end{exe}  

However, as \citet{BearthT2003} observes, word order in Bantu is typically not as much a question of syntactic restrictions as it is governed by the needs of discourse. Modifiers most commonly follow their head:

\begin{exe}
	\ex \gll ɪɪ-heela ny-ingi\\
	\textsc{aug}-money(10) 10-many\\
	\glt `much money'
	\ex \gll ʊ-tʊ-ndʊ tw-ɪtʊ\\
	\textsc{aug}-13-thing 13-\textsc{poss.1pl}\\
	\glt `our things'
\end{exe}

A key element in Nyakyusa syntax are the \isi{noun classes} (\sectref{NounClasses}) and their respective agreement or cross-reference. Agreement in Nyakyusa occurs between a head and its modifiers in the noun phrase, as well as between the predicate and its subject and possibly an object. There is also endophoric agreement with demonstratives and relative clauses.\is{subordinate clauses!relative clause} The reader is referred to \citet{KatambaF2003} and \citet{BearthT2003} for an introduction.

\label{RelativeClausss} 
\is{subordinate clauses!relative clause|(}
Unlike many other Bantu languages (see \citealt[ch. 3.3]{GueldemannT1996}), Nyakyusa does not have dedicated morphological paradigms for relative clauses. Instead these are introduced by demonstratives that agree with the head of the phrase. Typically the proximal or distal demonstratives are used, although emphatic reduplicated pronouns (see \sectref{Demonstratives}) are also found. (\ref{exSubjectRelativeClause}, \ref{exObjectRelativeClause}) illustrate subject and object relatives, respectively. (\ref{exLocativeRelativeClause}) exemplifies a relative clause modifying a locative adjunct.

\begin{exe}
	\ex \label{exSubjectRelativeClause} \gll ba-mo \textbf{a}-\textbf{ba} \textbf{ba}-\textbf{tol}-\textbf{iigwe} ʊ-kʊ-ly-ag-a ɪ-ly-ʊndʊ, ba-a-gelek-el-aga ɪ-mi-pʊpa \\
	2-one \textsc{aug}-2.\textsc{prox} 2--win-\textsc{pass.pfv} \textsc{aug}-15-5-find-\textsc{fv} \textsc{aug}-5-thatching\_grass 2-\textsc{pst}-thatch-\textsc{appl}-\textsc{ipfv} \textsc{aug}-4-banana\_leaf \\
	\glt `Some who were unable to get the type of grass would roof with banana leaves.' [Nyakyusa houses of long ago]
	\ex \label{exObjectRelativeClause}\gll leelo ʊ-lw-ifi bo lʊ-pɪliike \textbf{a}-\textbf{ma}-\textbf{syʊ} ga-a kalʊlʊ \textbf{a}-\textbf{ga} a-a-job-aga, lʊ-lɪnkw-amul-a lʊ-lɪnkʊ-tɪ\\
	now/but \textsc{aug}-11-chameleon as 11-hear.\textsc{pfv} \textsc{aug}-6-word 6-\textsc{assoc} hare(1) \textsc{aug}-\textsc{prox.6} 1-\textsc{pst}-speak-\textsc{ipfv} 11-\textsc{narr}-answer-\textsc{fv} 11-\textsc{narr}-say\\
	\glt `When Chameleon heard the words that Hare had been speaking, he answered:' [Hare and Chameleon]
	\ex \label{exLocativeRelativeClause} \gll a-lɪnkʊ-ga-ag-a a-m-ɪɪsi ma-nandɪ \textbf{n̩}-\textbf{dʊ}-\textbf{kʊbo} \textbf{mu}-\textbf{no} a-a-tiim-aga ɪɪ-ng'ombe\\
	1-\textsc{narr}-6-find-\textsc{fv} \textsc{aug}-6-water 6-little 18-11-pasture 18-\textsc{prox} 1-\textsc{pst}-herd-\textsc{ipfv} \textsc{aug}-cow(10)\\
	\glt `He found a little water in the pasture where he was herding cows.' [Water and toads] 
\end{exe}

Relative clauses referring to \isi{discourse participants} are introduced by their respective substitutives without the augment. (\ref{exRelativeClauseParticipants1}, \ref{exRelativeClauseParticipants2}) illustrate this for the first and second person singular.
\begin{exe}
	\ex \label{exRelativeClauseParticipants1}
	\gll n-dɪ na=a-ma-hala jʊ$\sim$jʊʊ-ne \textbf{ne} \textbf{n}-\textbf{ga}-\textbf{job}-\textbf{a} na=si-mo\\
	\textsc{1sg}-\textsc{cop} \textsc{com}=\textsc{aug}-6-intelligence \textsc{redupl}$\sim$1-\textsc{1sg} \textsc{1sg} \textsc{1sg}-\textsc{neg}-speak-\textsc{fv} \textsc{com}=10-one\\
	\glt \lq I'm the only smart one, I who haven't said anything.' [Invaders]
	\ex\label{exRelativeClauseParticipants2}\gll ʊ-ne n-gʊ-fi-tol-a n=ɪ-fi-nyamaana fi-f-ingi, aa=kʊ-j-a jo ʊ-gwe \textbf{gwe} \textbf{ʊ}-\textbf{bagiile} ʊ-kʊ-tolan-a na=niine?\\
	\textsc{aug}-\textsc{1sg} \textsc{1sg}-\textsc{prs}-8-beat-\textsc{fv} \textsc{com}=\textsc{aug}-8-animal 8-8-many \textsc{fut}=\textsc{prs}-be(come)-\textsc{fv} \textsc{ref.1} \textsc{aug}-\textsc{2sg} \textsc{2sg} \textsc{2sg}-be\_able.\textsc{pfv} \textsc{aug}-15-compete-\textsc{fv} \textsc{com}=\textsc{com.1sg}\\
	\glt `I beat [i.e. have beaten] many animals, will you be the one who is able to compete with me?' [Hare and Chameleon]
\end{exe}

Lastly, a relative clause need not have a lexical head. Endophoric reference and the inherent semantics of noun classes may substitute for this. (\ref{ExHeadlessRelativeClause}) illustrates a headless relative clause.
\begin{exe}
	\ex \label{ExHeadlessRelativeClause}
	\gll pa-lʊ-komaano lʊ-la, j-aa-sal-iigwe ɪɪ-fubu ʊ-kʊ-j-a mw-ɪmɪlɪli gw-a \textbf{ba}-\textbf{la} \textbf{bi}-\textbf{kʊ}-\textbf{j}-\textbf{a} pa-kʊ-tolan-a\\
	16-11-meeting 11-\textsc{dist} 9-\textsc{pst}-choose-\textsc{pass.pfv} \textsc{aug}-hippo(9) \textsc{aug}-15-be(come)-\textsc{fv} 1-supervisor 1-\textsc{assoc} 2-\textsc{dist} 2-\textsc{prs}-be(come)-\textsc{fv} 16-15-compete-\textsc{fv}\\
	\glt `At that meeting, Hippo was chosen as the referee of those that were going to race.' [Hare and Chameleon]
\end{exe}
\is{subordinate clauses!relative clause|)}
\is{syntax|)}