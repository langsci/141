\chapter{Tense and aspect constructions 3: futurates}
\label{Futurates} 
\section{Introduction}
\is{futurate|(}
It is a common feature of languages to have several constructions with future time orientation. This synchronic multiplicity of forms arises through historical layering of \isi{grammaticalization} processes from different sources as well as from similar sources at different periods (\citealt{BybeePagliucaPerkins1991}; \citealt{BybeePerkinsPaglucia1994}). Not all of these constructions can be considered to encode future tense\is{tense!future} in the sense of \sectref{TenseAspect}. Therefore \textsc{futurate} will be used as an umbrella term, following \citet{BinnickR1991}.

From a wider Bantu perspective, Nyakyusa is unusual in having more constructions for future states-of-affairs than for the past (cf. \citealt[89]{NurseD2008}). All of these are grammaticalized from two sources. On the one hand, there are paradigms featuring reflexes of \ili{Proto-Bantu} \textit{*gɪ̀} \lq go', which is found in the present day language as a movement\is{motion} gram (\sectref{jaAspectualizer}) as well as a \isi{copula} \lq be(come)' (see  \sectref{Copulae}). On the other hand, there are constructions featuring \textit{isa} \lq come' (\sectref{isaAspectualizer}). Leaving aside the copula, one finds a parallel distribution of the two movement verbs: both are found with their bare stems as proclitics\is{proclitic} to the inflected verb (\sectref{ProcliticAa}, \ref{ProcliticIsa}) and both are found in constructions that originate in their use as auxiliaries\is{auxiliary} in the \isi{simple present} (\sectref{Prospectivekwa}, \ref{isaFut}).

An intriguing feature of futurates in Nyakyusa is the possibility of combining them into a single complex predicate. The following examples illustrate these possibilities:
\begin{exe}
\ex \label{exFuturatesCombinations}
\begin{xlist}
\ex \gll tʊ-kw-a kʊ-kin-a\\
\textsc{1pl}-\textsc{prs}-go.\textsc{fv} 15-play-\textsc{fv}\\
\glt \lq We will go and play.'
\ex \gll aa=tʊ-kw-a kʊ-kin-a\\
\textsc{fut}=\textsc{1pl}-\textsc{prs}-go.\textsc{fv} 15-play-\textsc{fv}\\
\glt `We will go and play (at some distant time).'
\ex \gll tw-isakw-a kʊ-kin-a\\
\textsc{1pl}-\textsc{indef.fut}-go.\textsc{fv} 15-play-\textsc{fv}\\
\glt `We might go and play.'
\ex  \gll aa=tw-isakw-a kʊ-kin-a\\
\textsc{fut}=\textsc{1pl}-\textsc{indef.fut}-go.\textsc{fv} 15-play-\textsc{fv}\\
\glt `We might go and play (at some distant time).' [ET]
\end{xlist}
\end{exe}
As can be seen in (\ref{exFuturatesCombinations}), the meaning of these complex futurates arises through the composition of its constituent parts. In the following discussion, each futurate construction will thus be dealt with separately.

Lastly, it has often been noted (\citealt[103]{DahlOe1985} among others) that there is a close conceptual tie between future orientation and modality.\is{modality} In the case of Nyakyusa, this becomes clearest in the cases of constructions that are on the threshold between futurate and modality (\sectref{isaFut}, \ref{Desiderative}, \ref{ModalFuture}). These have been classified as either futurates or modals according to which meaning component prevails in the present-day language.
\is{futurate|)}
\section{Proclitic \textit{aa}=}\label{ProcliticAa}
\is{future!future aa=|(}\is{tense!future|(}
A clitic\is{proclitic} \textit{aa}= can be added to all constructions that have future readings, namely the \isi{simple present} (\sectref{Present}), the other futurates\is{futurate} (\sectref{Prospectivekwa}--\ref{ProspectiveKujapa}), the subjunctive\is{mood!subjunctive} (\sectref{Subjunctive}), the desiderative (\sectref{Desiderative})\is{mood!desiderative} and the modal future\is{future!modal future} construction (\sectref{Commissive}). It can also attach to \isi{conditional} \textit{ngalɪ} (\sectref{ngali}). Concerning the pre-initial position in Bantu, \citeauthor{GueldemannT2003} observes:
\begin{quote}
For pre-initial forms one can say that they are most likely the result of concatenation and truncation of a binary predicate structure. The first part is a finite auxiliary or non-finite predicator while the second part is a dependent finite form comprising the content verb. In the first case, the auxiliary is subject to phonetic truncation and the initial of the content verb continues to encode the subject.  \citep[186]{GueldemannT2003}
\end{quote}
The clitic's shape, its future semantics and its opposition to a \isi{proclitic} form of \textit{isa} `come' in some varieties of Nyakyusa (\sectref{ProcliticIsa}) all give evidence that the first part of this former binary structure included the \isi{auxiliary} (\textit{j})\textit{a} `go'.

As for its semantics, \textit{aa}= localizes the eventuality described in its host verb to a future reference frame -- or temporal domain, in \citeauthor{BotneRKershnerT2008}'s (\citeyear{BotneRKershnerT2008}) terminology. \citet[316]{NurseD2008} calls this kind of marker a ``shifter'', which is defined as ``a clitic, which, added to an existing tensed form, shifts its reference further away from the reference point''. This notion of shifting will become more tangible when looking at some uses of \textit{aa}=.

A very common collocation consists of \textit{aa}= and the simple present.\is{simple present} This collocation covers many of the prototypical uses ascribed to a future tense (\citealt{DahlOe1985}; \citeyear{DahlOe2000b}). It is the default form for pure predictions about the future (\ref{exAaPRSnoIntention1}, \ref{exAaPRSnoIntention2}) and can also be used when intention is involved (\ref{exAaPRSwithIntention1}, \ref{exAaPRSwithIntention2}).
 
\begin{exe}
\ex \label{exAaPRSnoIntention1} Context: What will happen if I eat this mushroom?\\
\gll lɪnga ʊ-l-iile, aa=kʊ-fw-a\\
if/when \textsc{2sg}-eat-\textsc{pfv} \textsc{fut}=\textsc{2sg.prs}-die-\textsc{fv}\\
\glt `If you eat it, you will die.' [ET]
\ex \label{exAaPRSnoIntention2} Context: It's no use trying to swim in the lake tomorrow.\\
\gll komma ʊ-kʊ-bʊʊk-a, a-m-ɪɪsi aa=gi-kʊ-j-a ma-talalɪfu\\
\textsc{proh} \textsc{aug}-15-go-\textsc{fv} \textsc{aug}-6-water \textsc{fut}=6-\textsc{prs}-be(come)-\textsc{fv} 6-cold\\
\glt `Don't go, the water will be cold.' [ET]
\ex \label{exAaPRSwithIntention1}Context: A young man is speaking about his plans for the future.\\
\gll lɪnga n-gʊl-ile a=n-gʊ-jeng-a ɪɪ-nyumba ɪɪ-nywamu\\
if/when \textsc{1sg}-grow-\textsc{pfv} \textsc{fut}=\textsc{1sg}-\textsc{prs}-build-\textsc{fv} \textsc{aug}-house(9) \textsc{aug}-big(9)\\
\glt `When I am grown up, I will build a big house.' [ET]
\ex \label{exAaPRSwithIntention2} Context: Talking about the speaker's plans for the evening.
\gll na=a-ma-jolo a=n-gʊ-j-a=po pa-ka-aja\\
\textsc{com}=\textsc{aug}-6-evening \textsc{fut}=\textsc{1sg}-\textsc{prs}-be(come)-\textsc{fv}=16 16-12-homestead\\
\glt `In the evening I will be at home.' [ET]
\end{exe}

Interestingly, Lusekelo (\citeyear{LusekeloA2007}; \citeyear{LusekeloA2013}) lists \textit{aa}= plus \isi{simple present} under ``future tense'' and states that ``there is only one future tense in Kinyakyusa'' \citep[109]{LusekeloA2013}, while a century earlier \citet[32]{SchumannK1899} listed it under ``rarely used constructions'' (Translated from the original German, BP) and \citet{EndemannC1914} did not discuss it at all. This coincides with the chronolect (and topolect)\is{dialects} described by the latter authors having the de-ventive indefinite future\is{future!indefinite future} (\sectref{isaFut}) as the primary construction for predictions. For a short discussion of the diachronic scenario see \sectref{isaFutDiachronic}. Also note that unlike the indefinite future,\is{future!indefinite future} the collocation of \textit{aa}= plus the \isi{simple present} does not allow for a purely epistemic\is{modality} reading without future time reference:
\clearpage
\begin{exe}
\ex Context: You are asked where your brother is.\\
\gll \#aa=i-kʊ-kin-a ʊ-m-pɪla\\
\phantom{\#}\textsc{fut}=1-\textsc{prs}-play-\textsc{fv} \textsc{aug}-3-ball\\
\glt \makebox[\myl][l]{}(intended: \lq He will be [=presumably he is] playing football.')
\ex Context: The cows are mooing. You are asked why they are making so much noise.\\
\gll \#aa=si-kʊ-j-a n=ɪ-n-jala\\
\phantom{\#}\textsc{fut}=10-\textsc{prs}-be(come)-\textsc{fv} \textsc{com}=\textsc{aug}-9-hunger\\
\glt \makebox[\myl][l]{}(intended: \lq They will be [=presumably they are] hungry.')
\end{exe} %myl is defined in localcommands.tex as the length of <#>. This is necessary, as the phantom commands do not work in a \glt line (they produce unwanted vertical space, even \hphantom does not

Some further examples from discourse will illustrate the shifting function of \textit{aa}=. In (\ref{exProcliticAaMonkeyTortoise}) Tortoise answers Monkeys's demand for payment by making an excuse and promising to pay at another time. Monkey in turn accepts and announces that he will come again. Both Tortoise's promise in (\ref{exProcliticAaMonkeyTortoiseA}) and Monkey's announcement in (\ref{exProcliticAaMonkeyTortoiseB}) feature the future proclitic,\is{proclitic} with the modal future\is{future!modal future} construction and the \isi{simple present} as the respective hosts. Omission of \textit{aa}= was judged inadequate in discussion of these examples. Note also how this contrasts with the use of the \isi{simple present} in (\ref{exPRSFutureMonkeTortoise}) on p.\nobreakspace\pageref{exPRSFutureMonkeTortoise}, which is taken from an earlier episode of the same narrative and describes Monkey's demand to be paid on that very day.
\begin{exe}
\ex Context: Monkey has come to Tortoise's in order to collect debts.
\label{exProcliticAaMonkeyTortoise}
\begin{xlist}
\ex \label{exProcliticAaMonkeyTortoiseA}
\gll po kajamba a-a-tɪ \textup{\lq\lq}hee. gʊʊ-hobok-el-ege. lɪlɪno n-dɪ n=ɪ-n-jɪla. n-sumwike kw-a ɗaaɗa gw-angʊ. a-lɪ n=ɪ-fy-ɪnja mia mooja. po a-bɪɪk-ile ʊ-bʊ-fumbwe\\
then tortoise(1) 1-\textsc{subsec}-say \phantom{\lq\lq}\textsc{interj} \textsc{2sg.1sg}-be(come)\_happy-\textsc{appl}-\textsc{ipfv.subj} now/today \textsc{1sg}-\textsc{cop} \textsc{com}=\textsc{aug}-9-path \textsc{1sg}-depart.\textsc{pfv} 17-\textsc{assoc} sister(1)(SWA) 1-\textsc{poss.1sg} 1-\textsc{cop} \textsc{com}=\textsc{aug}-8-year one(SWA) hundred(SWA). then 1-put-\textsc{pfv} \textsc{aug}-14-concern\\
\glt `Tortoise said ``Hee. Forgive me. Now I am travelling. I'm heading to my sister's. She is a hundred years old. She has made an invitation.'
\ex \gll po lee koo=kʊ-no n-gʊ-bʊʊk-a. po leelo ɪɪ-heela j-aako \textbf{aa}=\textbf{kw}-\textbf{eg}-\textbf{aga} kangɪ bo n-iis-ile\textup{''}\\
then now/but \textsc{ref.17}=17-\textsc{prox} \textsc{1sg}-\textsc{prs}-go-\textsc{fv} then now/but \textsc{aug}-money(9) 9-\textsc{pos.2sg} \textsc{fut}=\textsc{2sg.mod.fut}-take-\textsc{mod.fut} again as \textsc{1sg}-come-\textsc{pfv}\\
\glt  \lq\lq There I'm going. Your money you shall take when I've come back.''{}'
\ex \label{exProcliticAaMonkeyTortoiseB}
\gll po mwa=n-gambɪlɪ a-a-tɪ \textup{\lq\lq}ee, haya! mma po \textbf{a}=\textbf{n}-\textbf{gw}-\textbf{is}-\textbf{a} kangɪ\textup{''}. a-lɪnkʊ-bʊʊk-a kʊ-ka-aja\\
then matronym-9-monkey 1-\textsc{subsec}-say \phantom{\lq\lq}yes OK(<SWA) no then \textsc{fut}=\textsc{1sg}-\textsc{prs}-come-\textsc{fv} again 1-\textsc{narr}-go-\textsc{fv} 17-12-homestead\\
\glt `Mr. Monkey said ``Yes, OK. I'll come again.'' He went home.' [Monkey and Tortoise]
\end{xlist}
\end{exe}

In (\ref{exProcliticAaQuestionHareSpider}) the trickster Hare asks Spider's prospective wife, rhetorically, if she is willing to draw eight buckets of water each time her future husband bathes. The reference frame evoked here is the time of being married, which is not only situated in the future, but also constitutes an entirely new situation.

\begin{exe}
\ex \label{exProcliticAaQuestionHareSpider}\gll ee, a-lɪ na=go lwele kʊkʊtɪ kɪ-lʊndɪ ki-k-oog-a ɪ-n-dobo jɪ-mo. ʊ-gwe kʊ-lond-a ʊ-neg-ege ɪ-n-dobo lwele, bʊle \textbf{aa}=\textbf{kʊ}-\textbf{bwesy}-\textbf{a}?\\
yes, 1-\textsc{cop} \textsc{com}=\textsc{ref.6} eight every 7-leg 7-\textsc{prs}-bathe-\textsc{fv} \textsc{aug}-9-bucket 9-one \textsc{aug}-\textsc{2sg} \textsc{2sg.prs}-want-\textsc{fv} \textsc{2sg}-draw\_liquid-\textsc{ipfv.subj} \textsc{aug}-10-bucket eight \textsc{q} \textsc{fut}=\textsc{2sg.prs}-overcome-\textsc{fv}\\
\glt `Yes, it [Spider] has eight legs, every leg bathes in one bucket. You, do you want to draw eight buckets full, will you bear that?' [Hare and Spider]
\end{exe}

Similarly, in (\ref{exProcliticAavsPRS}) the speaker, an anthropomorphized python, first employs the \isi{simple present} as a generic\is{aspect!generic} futurate.\is{futurate} In the following sentence, \textit{aa}= serves to shift to a specific reference frame which is characterized by a change in conditions, namely that the children are locked in.

\begin{exe}
\ex \label{exProcliticAavsPRS}
Context: A python wants to devour a woman's children. She has announced that she will lock them in. Python answers.\\
\gll mma, n-gʊ-ba-ag-a. p-oope \textbf{a}=\textbf{n}-\textbf{gʊ}-\textbf{ba}-\textbf{ag}-\textbf{a}\\
no \textsc{1sg}-\textsc{prs}-2-find-\textsc{fv}. 16-also \textsc{fut}=\textsc{1sg}-\textsc{prs}-2-find-\textsc{fv}\\
\glt \lq No, I'm finding them (sic!). Even then I'll find them.' [Python and woman]
\end{exe}

The first sentence of (\ref{exProcliticAavsPRS}) illustrates another important fact about the organization of the future in Nyakyusa. While the collocation of \textit{aa}= with the \isi{simple present} is the common form for making predictions, the bare simple present may be employed to convey a high degree of certainty. Thus (\ref{exPRSFuturateVsProclitic}) contrasts with (\ref{exAaPRSnoIntention1}) above in its epistemic\is{modality} value. See also (\ref{exPRSfuturate}) on p.\nobreakspace\pageref{exPRSfuturate}.

\begin{exe}
\ex
\label{exPRSFuturateVsProclitic}
Context: What will happen if I eat this mushroom?\\
\gll lɪnga ʊ-l-iile, kʊ-fw-a\\
if/when \textsc{2sg}-eat-\textsc{pfv} \textsc{2sg.prs}-die-\textsc{fv}\\
\glt \lq If you eat it, you die (i.e. this is a known fact).' [ET]
\end{exe}

The employment of \textit{aa}= with a number of other future-orientated paradigms is illustrated in the following examples. In (\ref{exProcliticAaSubjunctiveKapyung}), the subjunctive\is{mood!subjunctive} verb carries \textit{aa}= because the reference frame is shifted to the time after the chief's death. (\ref{exProcliticAaProgressive}--\ref{exProcliticAaIndefFut}) illustrate the use of \textit{aa}= with the periphrastic progressive,\is{aspect!progressive}\footnote{While the present and past progressives feature the copula \textit{lɪ}, for future reference \textit{ja} \lq be(come)' has to be used. See \sectref{Copulae} on the two copulae.\is{copula}} the de-itive prospective/movement\is{motion} construction and the indefinite future,\is{future!indefinite future} respectively. For an example featuring the desiderative,\is{mood!desiderative} see (\ref{exDesiderativeHareHippo}) on p.\nobreakspace\pageref{exDesiderativeHareHippo}.
\begin{exe}
\ex \label{exProcliticAaSubjunctiveKapyung}
Context: A moribund chief gives his sons instructions for the time after his death.\\
 \gll looli ɪ-si n-gʊ-ba-bʊʊl-a, \textbf{aa}=\textbf{mu}-\textbf{si}-\textbf{kol}-\textbf{ege} fiijo lɪnga m-fw-ile\\
but \textsc{aug}-\textsc{prox.10} \textsc{1sg}-\textsc{prs}-\textsc{2pl}-tell-\textsc{fv} \textsc{fut}=\textsc{2pl}-10-grasp-\textsc{ipfv.subj} \textsc{intens} if/when \textsc{1sg}-die-\textsc{pfv}\\
\glt `The things I tell you, you must stick to when I'm dead.' [Chief Kapyungu]
\ex \label{exProcliticAaProgressive} \gll \textbf{aa}=\textbf{i}-\textbf{kʊ}-\textbf{j}-\textbf{a} pa-kʊ-jeng-a\\
\textsc{fut}=1-\textsc{prs}-be(come)-\textsc{fv} 16-15-build-\textsc{fv}\\
\glt \lq He will be building.' [ET]
\ex \gll \textbf{aa}=\textbf{tʊ}-\textbf{kw}-\textbf{a} kʊ-kin-a ʊ-m-pɪla\\
\textsc{fut}=\textsc{1pl}-\textsc{prs}-go.\textsc{fv} 15-play-\textsc{fv} \textsc{aug}-3-ball\\
\glt \lq We will go and play football (at some distant time).' [ET]
\ex  \label{exProcliticAaIndefFut} \gll pa-lʊ-bʊnjʊ \textbf{aa}=\textbf{tw}-\textbf{isakʊ}-\textbf{kin}-\textbf{a} ʊ-m-pɪla\\
16-11-morning \textsc{fut}=\textsc{1pl}-\textsc{indef.fut}-play-\textsc{fv} \textsc{aug}-3-ball\\
\glt \lq The day after tomorrow we might play football.' [ET]
\end{exe}

The future \isi{proclitic} can also attach to a \isi{simple present} that stands as the complement of the persistive aspect\is{aspect!persistive} \isi{auxiliary} (\sectref{Persistive}). This is illustrated in (\ref{exProcliticAaPersisitve}). Lastly, for a discussion of \textit{aa}= together with the \isi{conditional} marker \textit{ngalɪ}, see \sectref{ngali}.
\begin{exe}
\ex \label{exProcliticAaPersisitve}\gll na=a-ma-jolo \textbf{ba}-\textbf{kaalɪ} \textbf{aa}=\textbf{bi}-\textbf{kʊ}-\textbf{mog}-\textbf{a}\\
\textsc{com}=\textsc{aug}-6-evening 2-\textsc{pers} \textsc{fut}=2-\textsc{prs}-dance-\textsc{fv}\\
\glt `In the evening they will still be dancing.' [ET]
\end{exe}
\is{future!future aa=|)}\is{tense!future|)}
\section{Proclitic (\textit{i})\textit{sa}=}\label{ProcliticIsa}
\is{future!future (i)sa=|(}
Parallel to the de-itive future proclitic \textit{aa=} (\sectref{ProcliticAa}), some varieties\is{dialects} of Nyakyusa feature a de-ventive clitic \textit{(i)sa=}. Though this form is not found in the varieties on which this description is based, it merits a short discussion for the sake of completeness.

\citet[31]{SchumannK1899} and \citet[54]{EndemannC1914} both list the combination of a proclitic \mbox{\textit{isa}=} and the \isi{simple present} without further discussions of meaning and use. Two tokens of this construction are found in the text collection by \citet{BergerP1933}, one of which is given in (\ref{exProcliticIsa1}). All three sources are based on the topolect\is{dialects} (and respective chronolect) of the lake-shore plains.

\begin{exe}
\ex \label{exProcliticIsa1}
Context: a discussion of how a group of people may cross a river.\\
 \gll leelo ʊ-ne ni-kʊ-laalʊʊsy-a ni-kʊ-tɪ: kalɪ b-isakʊ-lobok-a bʊle$\sim$bʊle pa-lw-ɪsi? namanga ʊ-lw-ɪsi lʊlʊʊ$\sim$lʊ kɪ-siba, kangɪ ɪ-n-gwina ny-ingi fiijo; mw-ene mu-m̩-bwato mo bi-kʊ-sʊʊbɪl-a ʊ-kw-end-a\\
now/but \textsc{aug}-\textsc{1sg} \textsc{1sg}-\textsc{prs}-ask-\textsc{fv} \textsc{1sg}-\textsc{prs}-say \textsc{q} 2-\textsc{indef.fut}-cross\_over\_water-\textsc{fv} \textsc{redupl}$\sim$how 16-11-river because \textsc{aug}-11-river \textsc{redupl}$\sim$\textsc{prox.11} 7-pond again \textsc{aug}-10-crocodile 10-many \textsc{intens} 18-only 18-18-boat(1)(<EN) \textsc{ref.18} 2-\textsc{prs}-hope-\textsc{fv} \textsc{aug}-15-walk/travel-\textsc{fv}\\
\glt \lq Now I ask and say: How will they cross the river? Because this river is very deep and the crocodiles are many. Only in a boat they can travel.'
\sn \gll  bʊle gwe mw-inangʊ, gwe kʊ-bal-a ɪɪ-nongwa sisii$\sim$si, kʊ-tɪ fi-ki, kw-inogon-a \textbf{sa}=\textbf{bi}-\textbf{kʊ}-\textbf{lobok}-\textbf{a} bʊle$\sim$bʊle?\\
\textsc{q} \textsc{2sg} 1-my\_companion \textsc{2sg} 15-read-\textsc{fv} \textsc{aug}-issue(10) \textsc{redupl}$\sim$\textsc{prox.10} \textsc{2sg.prs}-say 8-what \textsc{2sg.prs}-think-\textsc{fv} come=2-\textsc{prs}-cross\_over\_water-\textsc{fv} \textsc{redupl}$\sim$how\\
\glt \lq  You, my friend, who you are reading these words, how do you think that they will cross over?' (\citealt[150]{BergerP1933};  orthography adapted)
% \sn used to keep at least the free translation together; anotheer option might be to employ \sn right after kw-inogon-a, in order to keep the last line in the text + gloss + free translation together (but this would create more white space at the bottom of the page)
\end{exe}

In a current draft of a Bible translation by SIL International, which is also based on the lake-shore-plains variety,\is{dialects} the \isi{proclitic} is found with host verbs inflected for the simple present,\is{simple present} the modal future\is{future!modal future} construction (\sectref{Commissive}) and the subjunctive\is{mood!subjunctive} (\sectref{Subjunctive}). Example (\ref{exProcliticIsa2}) again features a host verb inflected for the simple present.\is{simple present} Note that, unlike what these examples may suggest, the \isi{proclitic} is found in both questions and declarative sentences. 

\begin{exe}
\ex \label{exProcliticIsa2}
Context: Jesus speaks about the Kingdom of Heaven.\\
\gll ʊ-n-twa a-lɪnkʊ-job-a a-lɪnkʊ-tɪ \lq\lq mu-sy-ag-an-i-e ɪ-si i-kʊ-job-a ʊ-n̩-dongi ʊ-n-niongafu!'' bʊle, Kyala a-ti-kʊ-ba-longel-a kanunu a-ba-sʊngʊligwa ba-ake, a-ba bi-kʊ-n̩-dɪlɪl-a pa-muu-si na pa-kɪ-lo? bʊle, \textbf{isa}=\textbf{i}-\textbf{kʊ}-\textbf{kaabɪl}-\textbf{a} ʊ-kʊ-ba-tʊʊl-a?\\
\textsc{aug}-1-lord 1-\textsc{narr}-speak-\textsc{fv} 1-\textsc{narr}-say \phantom{\lq\lq}\textsc{2pl}-10-find-\textsc{recp}-\textsc{caus}-\textsc{subj} \textsc{aug}-\textsc{prox.10} 1-\textsc{prs}-speak-\textsc{fv} \textsc{aug}-1-judge \textsc{aug}-1-twisted \textsc{q} God(1) 1-\textsc{neg}-\textsc{prs}-2-judge-\textsc{fv} well \textsc{aug}-2-selected 2-\textsc{poss.sg} \textsc{aug}-\textsc{prox.2} 2-\textsc{prs}-1-lament-\textsc{fv} 16-3-daytime \textsc{com} 16-7-night \textsc{q} come=1-\textsc{prs}-be\_late-\textsc{fv} \textsc{aug}-15-2-help-\textsc{fv}?\\
\glt \lq And the Lord said, Hear what the unjust judge saith. And shall no God avenge his own elect, which cry day and night unto him, though he bear long with them?'
\sn \gll n-gʊ-ba-bʊʊl-a ʊkʊtɪ, Kyala i-kʊ-ba-longel-a m̩bɪbɪ$\sim$m̩bɪbɪ ɪ-fi bi-kʊ-lond-igw-a ʊ-kʊ-kab-a. looli bo i-kw-is-a n-nya-mu-ndʊ, bʊle, \textbf{isa}=\textbf{i}-\textbf{kʊ}-\textbf{ba}-\textbf{ag}-\textbf{a} a-ba-ndʊ a-ba bi-kʊ-mmw-itɪk-a?\\
\textsc{1sg}-\textsc{prs}-\textsc{2pl}-tell-\textsc{fv} \textsc{comp} God 1-\textsc{prs}-2-judge-\textsc{fv} \textsc{redupl}$\sim$fast \textsc{aug}-\textsc{prox.8} 2-\textsc{prs}-want-\textsc{pass}-\textsc{fv} \textsc{aug}-15-get-\textsc{fv} but as 1-\textsc{prs}-come-\textsc{fv} 1-kinship-1-person \textsc{q} come=1-\textsc{prs}-2-find-\textsc{fv} \textsc{aug}-2-person \textsc{aug}-\textsc{prox.2} 2-\textsc{prs}-1-believe-\textsc{fv}\\
\glt \lq  I tell you that he will avenge them speedily. Nevertheless when the Son of man cometh, shall he find faith on earth?' (Luke 18: 6--8)
\end{exe}

In elicitation, all speakers consulted for the present study rejected this kind of construction. When presented with examples from \citet{BergerP1933} and from the translation of the Bible, they similarly considered these to be erroneous. Some would \lq\lq correct'' them to feature the indefinite future\is{future!indefinite future} (\sectref{isaFut}).

Given that the de-ventive \isi{proclitic} does not occur in the varieties\is{dialects} in focus here, a functional characterization clearly lies outside the scope of the present study. In a first examination of its uses in the Bible translation a vague pattern, however, emerges, in which the \isi{proclitic} is mostly used in the last verb in a paragraph, more often than not in the context of prophecies. Thus both tokens in (\ref{exProcliticIsa2}) do not only deal with judgement day, but also close the respective utterances of God and Jesus. This observation must, of course, be taken with a grain of salt, given the peculiarities of religious material. But it does fit the fact that in (\ref{exProcliticIsa1}), the author, after an exposition of the subjects' situation, turns towards the reader to ask for his evaluation of how their dilemma will eventually be solved.
\is{future!future (i)sa=|)}
\section{Proclitic \textit{naa}=}\label{ProcliticNaa}\is{future!future naa=|(}
In the southern topolects of Nyakyusa, a further proclitic \textit{naa}= with a future or \isi{futurate} meaning is found. Its shape indicates that this proclitic might be a portmanteau of comitative \textit{na} and the future\is{tense!future} \isi{proclitic} \textit{aa}=\is{future!future aa=} (\sectref{ProcliticAa}). See p.\nobreakspace\pageref{SubjunctiveNa} in \sectref{SubjunctiveNa} for \textit{na} together with the subjunctive mood.\is{mood!subjunctive} The only token in the text corpus is (\ref{exProclitNaaInLaw}), which is taken from a \isi{narrative} told by a speaker of the lake-shore-plains varieties.\is{dialects} See also (\ref{exProclitNaaRuth}), which is based on the same variety.

In elicitation, speakers of the Selya variety knew this construction, although the difference between this and \isi{proclitic} \textit{aa}=\is{future!future aa=} remains unclear. Speakers from the village of Lwangwa -- at the transition between Selya to the south and Mwamba (Lugulu) to the north (\sectref{Varietydescribed}) --
and speakers of the Mwamba/Lugulu variety either rejected it or considered it a feature of more southern varieties.\is{dialects}

\begin{exe}
\ex \label{exProclitNaaInLaw} \gll po p-ii-balasi j-aa-j-a=po ɪɪ-meesa a-pa paa$\sim$pa N̩goosi a-a-tʊʊl-ile ʊ-bʊ-fu bw-ake, bʊ-la ba-j-ile kʊ-sy-a, ʊ-bʊ \textbf{naa}=\textbf{i}-\textbf{kʊ}-\textbf{bʊʊk}-\textbf{a} na=bo ʊ-gwise\\
then 16-5-veranda(<SWA) 9-\textsc{subsec}-be(come)-\textsc{fv}=16 \textsc{aug}-table(9)(<SWA) \textsc{aug}-\textsc{prox.16} \textsc{redupl}$\sim$\textsc{prox.16} N. 1-\textsc{pst}-take\_from\_head-\textsc{pfv} \textsc{aug}-14-flour 14-\textsc{poss.sg} 16-\textsc{dist} 2-go-\textsc{pfv} 15-grind-\textsc{fv} \textsc{aug}-\textsc{prox.14} \textsc{fut}=1-\textsc{prs}-go-\textsc{fv} \textsc{com}=\textsc{ref.14} \textsc{aug}-his\_father(1)\\
\glt \lq On the veranda there was a table where Ngoosi had put the flour that they had gone to grind, the one that her father would go with.' [Man and his in-law] %bessere Übersetzung -> FleX

\ex \label{exProclitNaaRuth}\gll po ʊ-mu-ndʊ jʊ-la ʊ-gw-a kɪ-lɪngo a-lɪnkw-amul-a, a-lɪnkʊ-tɪ \textup{\lq\lq}a-pa bo si-lɪ bʊno$\sim$bʊ-no, ʊ-ne n-dek-ile, n-ga-bagɪl-a ʊ-kʊ-ʊl-a paapo \textbf{na}=\textbf{n}-\textbf{gw}-\textbf{is}-\textbf{a} k-oonang-a ɪ-kɪ-lɪngo ky-a ba-anangʊ a-ba-a kw-is-a kw-ingɪl-a=po pa-my-angʊ\textup{''}\\
then \textsc{aug}-1-person 1-\textsc{dist} \textsc{aug}-1-\textsc{assoc} 7-inheritance 1-\textsc{narr}-answer-\textsc{fv} 1-\textsc{narr}-say \phantom{\lq\lq}\textsc{aug}-\textsc{prox.16} as 10-\textsc{cop} \textsc{redupl}$\sim$14-\textsc{dem} \textsc{aug}-\textsc{1sg} \textsc{1sg}-let-\textsc{pfv} \textsc{1sg}-\textsc{neg}-be\_able-\textsc{fv} \textsc{aug}-15-buy-\textsc{fv} because \textsc{fut}=\textsc{1sg}-\textsc{prs}-come-\textsc{fv} 15-destroy-\textsc{fv} \textsc{aug}-7-inheritance 7-\textsc{assoc} 2-my\_child \textsc{aug}-2-\textsc{assoc} 15-come-\textsc{fv} 15-enter-\textsc{fv}=16 16-4-\textsc{poss.1sg}\\
\glt \lq And the kinsman said \lq\lq I cannot redeem it for myself, lest I mar mine own inheritance [lit. \ldots  I cannot buy it because I will come to destroy the inheritance of my children who will come to succeed me]') (Ruth 4:6)

\end{exe}
\is{future!future naa=|)}

\section{Prospective/movement \textit{kwa} INF}
\is{futurate|(}\is{aspect!prospective|(}\is{motion|(}
\label{Prospectivekwa}
The simple present form of the movement gram \textit{(j)a} (\sectref{jaAspectualizer}), together with an augmentless infinitive,\is{infinitive} serves as a marker of prospective aspect.

\begin{exe}
\ex \textit{tʊkwa kʊjoba} \lq we are going to (go and) speak'
\end{exe}

As \citet[76]{ComrieB1976} defines it, prospective aspect expresses that ``a state is related to some subsequent action, for instance when someone is in a state of being about to do something''. This is further refined by \citet[18f]{FleischmanS1982}, who characterizes \lq go-futures' as having a set of closely related and partly overlapping uses, denoting present relevance, imminence, intentionality, inception or assumed eventualities. In Nyakyusa, there are two constructions that each cover some of these uses: the de-itive constuction discussed in this section and the prospective/inceptive construction discussed in \sectref{ProspectiveKujapa}.

The following examples will illustrate the use of prospective/movement \textit{kwa}. In (\ref{exkwakuThievingMonkeys}), the first occurrence of the construction denotes the speaker's intention to throw pepper at the thieving monkeys. Apart from intentionality, the notion of movement remains intact. The second occurrence presents the assumed reaction of those monkeys, which is also situated away from the speaker's current location. Similarly, in (\ref{exkwakuHareSpider}) Hare begs local people to help him descend from a tree where he is trapped, thus present relevance, and presents his intended action. Again, the original sense of the movement gram remains intact, as Hare's action is situated away from his initial location. Lastly, in (\ref{exkwakuHareHippo}) Hare declares his coming back to a group of girls who have just rebuffed him, thus there is an overlap between intentionality and present relevance.
 
\begin{exe}
\ex \label{exkwakuThievingMonkeys} \gll tʊ-tik-e ɪ-m-bilipili tʊ-bɪɪk-e n-tʊ-supa. \textbf{tʊ}-\textbf{kw}-\textbf{a} kʊ-si-sop-el-a paapo ɪ-sy-ene \textbf{si}-\textbf{kw}-\textbf{a} kʊ-t-ɪgɪ \textup{\lq\lq}bi-kʊ-tʊ-p-a ɪ-fi-ndʊ\textup{''}\\
\textsc{1pl}-pound-\textsc{subj} \textsc{aug}-9-pepper \textsc{1pl}-put-\textsc{subj} 18-13-bottle \textsc{1pl}-\textsc{prs}-go.\textsc{fv} 15-10-throw-\textsc{appl}-\textsc{fv} because \textsc{aug}-10-self 10-\textsc{prs}-go.\textsc{fv} 15-say-\textsc{ipfv} \phantom{\lq\lq}2-\textsc{prs}-\textsc{1pl}-give-\textsc{fv} \textsc{aug}-8-food\\
\glt `We should pound pepper and put it in little bottles. Then we will go throw them at them [monkeys], for they'll think ``they're throwing food.''{}' (Thieving Monkeys)%1. intention Bewegungskomponente dabei 2.assumed und IPFV

\ex \label{exkwakuHareSpider}
\gll n-gʊ-sʊʊm-a mw-eg-e ʊ-lʊ-goje, mu-m-biny-e ɪ-m-bʊlʊkʊtʊ. mu-kol-e fiijo, mu-sulusy-ege panandɪ$\sim$panandɪ. lɪnga m-fik-ile \textbf{n}-\textbf{gw}-\textbf{a} kʊ-jɪgɪsy-a ʊ-lʊ-goje, ʊ-mwe mu-kʊ-lek-esy-aga\\
\textsc{1sg}-\textsc{prs}-beg-\textsc{fv} \textsc{2pl}-take-\textsc{subj} \textsc{aug}-11-rope \textsc{2pl}-\textsc{1sg}-bind-\textsc{subj} \textsc{aug}-10-ear \textsc{1pl}-grasp-\textsc{subj} \textsc{intens} \textsc{2pl}-lower-\textsc{ipfv.subj} \textsc{redupl}$\sim$a\_little if/when \textsc{1sg}-arrive-\textsc{pfv} \textsc{1sg}-\textsc{prs}-go.\textsc{fv} 15-shake-\textsc{fv} \textsc{aug}-11-rope  \textsc{aug}-\textsc{2pl} \textsc{2pl}-\textsc{mod.fut}-let-\textsc{caus}-\textsc{mod.fut}\\
\glt `I beg you (pl.) to take a rope and tie it to my ears. Hold it tight, lower it step by step. When I arrive [at the ground] I will shake the rope and you shall let go of it.' [Hare and Spider] %intention, present relevance%bessere Übersetzung -> FleX

\ex \label{exkwakuHareHippo}\gll \textbf{n}-\textbf{gw}-\textbf{a} kw-is-a n=ʊ-n-kamu gw-angʊ ʊ-n-nywamu fiijo\\
\textsc{1sg}-\textsc{prs}-go.\textsc{fv} 15-come-\textsc{fv} \textsc{com}=\textsc{aug}-1-relative 1-\textsc{poss.1sg} \textsc{aug}-1-big \textsc{intens}\\
\glt `I will come with my very big relative.' [Hare and Hippo] %intention
\end{exe}

Note that the \isi{simple present} form of (\textit{j})\textit{a} only serves this function as a prospective/movement futurate. It does not have a habitual/generic\is{aspect!habitual}\is{aspect!generic} reading (see \sectref{jaAspectualizer}), nor does it have a progressive\is{aspect!progressive} one and nor does it denote motion with purpose:

\begin{exe}
\ex \label{exKwaNotProg} \gll n-gw-a kʊ-kin-a ʊ-m-pɪla\\
\textsc{1sg}-\textsc{prs}-go.\textsc{fv} 15-play-\textsc{fv} \textsc{aug}-3-ball\\
\glt \lq I will (go and) play football.'\\
not: \lq I am going (in order) to play football.'\\
not: \lq I am going to (the place that we) play football.'
\end{exe}

In its use as a futurate, the complement of (\textit{j})\textit{a} can take the imperfective\is{aspect!imperfective} final suffix \mbox{-\textit{aga}}. To begin with, this gives a continuous/progressive reading, which can shade into an emphatic one (\ref{exkwakuAgaFootball}). The imperfective can also add the epistemic\is{modality} flavour of an assumed eventuality, as in (\ref{exkwakuAgaBuild}) and in the second token in (\ref{exkwakuThievingMonkeys}) above. Lastly, -\textit{aga} can also give a habitual/generic reading (\ref{exkwakuAgaDance}).\is{aspect!habitual}\is{aspect!generic}

\begin{exe}
\ex \label{exkwakuAgaFootball}
\gll lɪlɪno tʊ-kw-a kʊ-kin-aga ʊ-m-pɪla\\
now/today \textsc{1pl}-\textsc{prs}-go.\textsc{fv} 15-play-\textsc{ipfv} \textsc{aug}-3-ball\\
\glt 1. \lq Today we will (go and) be playing football (i.e. not stop).'\\
2. \lq Today we will (go and) be playing (no matter what).' [ET]

\ex \label{exkwakuAgaBuild}
\gll i-kw-a kʊ-jeng-aga kʊ-Tʊkʊjʊ\\
1-\textsc{prs}-go.\textsc{fv} 15-build-\textsc{ipfv} 17-T.\\
\glt 1. \lq He will (go and) be building in Tukuyu (continuously).'\\
2. \lq He will (go and) build in Tukuyu (assumedly).' [ET]

\ex \label{exkwakuAgaDance}
 \gll kʊkʊtɪ ky-ɪnja bi-kw-a kʊ-mog-aga ii-ng'oma\\
every 7-year 2-\textsc{prs}-go.\textsc{fv} 15-dance-\textsc{ipfv} 5-type\_of\_dance\\
\glt \lq Every year they will go and dance the \textit{ng'oma} dance.' [ET]
\end{exe}
\is{futurate|)}\is{aspect!prospective|)}\is{motion|)}
\section{Indefinite future}\label{isaFut}
\is{future!indefinite future|(}
\subsection{Formal makeup}
The indefinite future consists of a pre-initial prefix \textit{isakʊ} and the final vowel -\textit{a} or imperfective\is{aspect!imperfective} -\textit{aga}.

\begin{exe}
\ex \textit{twisakʊjoba} \lq we will possibly speak'
\end{exe}

The indefinite future constitutes an advanced stage of \isi{grammaticalization} of \textit{isa} \lq come' as an \isi{auxiliary} in the \isi{simple present} (\sectref{isaAspectualizer}). While its source construction is still mostly transparent, no material can intervene between what corresponds to the original auxiliary and its \isi{infinitive} complement. 

\begin{exe}
\ex \begin{xlist}
\ex[]{\gll tw-isakʊ-kin-a ʊ-m-pɪla m̩-bʊ-sikʊ bʊla$\sim$bʊ-la\\
\textsc{1pl}-\textsc{indef.fut}-play-\textsc{fv} \textsc{aug}-3-ball 18-14-time \textsc{redupl}$\sim$14-\textsc{dist}\\
\glt \lq We might play football some day.' [ET]
}
\ex[*]{twisa m̩bʊsikʊ bʊlabʊla kʊkina ʊmpɪla}
\end{xlist}
\end{exe}

Furthermore, juxtaposition with the subject prefix yields the expected output concerning glide formation and vowel quality (\sectref{HiatusSolution}), but does not result in a long vowel\is{vowels!length} (\ref{exIsaFutVowelShortNCL}). This also holds for the prefix of the first person singular\is{subject marker} (\ref{exIsaFutVowelShort1sg}).
\begin{exe}
\ex \label{exIsaFutVowelShortNCL}\begin{tabbing}
\textit{sisakʊjoba}x\=(\degree si-isakʊ-job-a)x\=\kill%unsinnszeile für Tabulatoren
\textit{isakʊjoba}\>(\degree a-isakʊ-job-a)\>`s/he will possibly speak'\\ 
\textit{bisakʊjoba}\>(\degree ba-isakʊ-job-a)\>`they will possibly speak'\\
\textit{jisakʊjoba}\>(\degree jɪ-isakʊ-job-a)\>`it (9) will possibly speak'\\ 
\textit{sisakʊjoba}\>(\degree si-isakʊ-job-a)\>`they (10) will possibly speak'
\end{tabbing} 
\ex\label{exIsaFutVowelShort1sg}
\begin{tabbing}
\textit{sisakʊjoba}x\=(\degree si-isakʊ-job-a)x\=\kill%unsinnszeile für Tabulatoren
\textit{nisakʊjoba}\>(\degree n-isakʊ-job-a)\>`I will possibly speak'
\end{tabbing}
\end{exe}

The indefinite future is negated with the \isi{negative} prefix \textit{t}(\textit{i}) following the subject marker.\is{subject marker} As the indefinite future is derived from the \isi{simple present} of \isi{auxiliary} \textit{isa}, it is assumed that the underlying negation is \textit{ti} (\sectref{NegPresent}). Again, the vowel remains short\is{vowels!length} (\ref{exNegativeIndefFuture}):

\begin{exe}
\ex \label{exNegativeIndefFuture}\textit{tʊtisakʊjoba} \lq we will not possibly speak' 
\end{exe}

In the following discussion, the meaning and uses of the indefinite future will be described as they are found synchronically in the varieties\is{dialects} that are in the focus of this study. After that, a diachronic perspective will be applied, which will help us to understand the position of the construction in question in the present-day Nyakyusa TMA system.

\subsection{Meaning and use}
\is{modality|(}\is{tense!future|(}
While older grammatical sketches consider the construction in question Nyakyusa's main future tense, it was hardly ever spontaneously offered by speakers in this study. In elicitation, language assistants unanimously rejected the use of this construction for prototypical predictions (\ref{exIsaFutPrediction1}, \ref{exIsaFutPrediction2}) or intention-based future eventualities (\ref{exIsaFutIntention1}, \ref{exIsaFutIntention2}). In these typical future tense contexts, the collocation of \textit{aa}=\is{future!future aa=} and the \isi{simple present} would be used; see (\ref{exAaPRSnoIntention1}--\ref{exAaPRSwithIntention2}) on p.\nobreakspace\pageref{exAaPRSnoIntention1}. The examples are based on \citeauthor{DahlOe1985}'s (\citeyear{DahlOe1985}; \citeyear{DahlOe2000b}) questionnaires.
\begin{exe}
\ex \label{exIsaFutPrediction1}
Context: What will happen if I eat this mushroom? -- You will die.\\
\gll \#lɪnga ʊ-l-iile gw-isakʊ-fw-a\\
\phantom{\#}if/when \textsc{2sg}-eat-\textsc{pfv} \textsc{2sg}-\textsc{indef.fut}-die-\textsc{fv}\\

\ex \label{exIsaFutPrediction2}
Context: It's no use trying to swim in the lake tomorrow. The water will be cold.\\
\gll \#a-m-ɪɪsi g-isakʊ-j-a ma-talalɪfu\\
\phantom{\#}\textsc{aug}-6-water 6-\textsc{indef.fut}-be(come)-\textsc{fv} 6-cold\\
\ex \label{exIsaFutIntention1}
Context: A young man's plans for the future. He intends to build a big house then.\\
\gll \#lɪnga n-gʊl-ile n-isakʊ-jeng-a ɪɪ-nyumba ɪɪ-nywamu\\
\phantom{\#}if/when \textsc{1sg}-grow-\textsc{pfv} \textsc{1sg}-\textsc{indef.fut}-build-\textsc{fv} \textsc{aug}-house(9) \textsc{aug}-big(9)\\
\ex \label{exIsaFutIntention2}
Context: Talking about the speaker's plans for the evening. He will be at home.\\
\gll \#n-isakʊ-j-a itolo pa-ka-aja\\
\phantom{\#}\textsc{1sg}-\textsc{indef.fut}-be(come)-\textsc{fv} just 16-12-homestead\\
\end{exe}

Examples containing the indefinite future that were constructed by the researcher were interpreted as describing events that are less probable and/or contingent on other circumstances.

\begin{exe}
\ex\gll tw-isakʊ-kin-a ʊ-m-pɪla\\
\textsc{1pl}-\textsc{indef.fut}-play-\textsc{fv} \textsc{aug}-3-ball\\
\glt \lq We might play football (unsure or dependent on circumstances).' [ET]
\ex\gll tw-isakʊ-fik-a kʊ-ka-aja k-ɪɪnʊ\\
\textsc{1pl}-\textsc{indef.fut}-arrive-\textsc{fv} 17-12-homestead 12-\textsc{poss.2pl}\\
\glt \lq We might arrive at your place some day (i.e do not overdo your bragging about it, we might come and check).' [ET]
\end{exe}

Examples (\ref{exIndefiniteFutureBySpeakers1}, \ref{exIndefiniteFutureBySpeakers2}) are typical of the few cases in which speakers themselves offered uses of the indefinite future within the wider elicitation context. Again, a future-oriented epistemic reading lies at the heart of these examples, apparently with some persistent ingressive flavour.

\begin{exe}
\ex \label{exIndefiniteFutureBySpeakers1} \gll mu-nga-bʊʊk-aga kʊ-m-ɪɪsi mwibeene, ɪ-n-joka \textbf{j}-\textbf{isakʊ}-\textbf{ba}-\textbf{mil}-\textbf{a}\\
\textsc{2pl}-\textsc{neg.subj}-go-\textsc{ipfv} 17-6-water \textsc{2pl}.self \textsc{aug}-9-snake 9-\textsc{indef.fut}-\textsc{2pl}-swallow-\textsc{fv}\\
\glt \lq Don't go to the water by yourselves, a snake might devour you.' [ET]
\ex \label{exIndefiniteFutureBySpeakers2} \gll a-ka-pɪɪj-a=mo ɪ-m-balaga, kangɪ \textbf{a}-\textbf{t}-\textbf{isakʊ}-\textbf{pɪɪj}-\textbf{a}=\textbf{mo}\\
1-\textsc{neg}-cook-\textsc{fv}=some \textsc{aug}-9-banana\_stew again 1-\textsc{neg}-\textsc{indef.fut}-cook-\textsc{fv}=some\\
\glt `She has never cooked banana stew and she won't likely ever do so.' [ET]
\end{exe}

Concerning uses of the indefinite future in a more spontaneous context, consider the following example. The researcher was standing on a path leading away from the village of Lwangwa and having a chat with some of the local inhabitants. A motorcycle taxi came rushing by and the researcher only jumped aside in the last moment. After it had passed, (\ref{exIsaFutBodaBoda}) was uttered, giving the researcher to understand that such unalert behaviour might at some point lead to an accident.

\begin{exe}
\ex \label{exIsaFutBodaBoda} \gll b-isakʊ-kʊ-tik-a\\
2-\textsc{indef.fut}-\textsc{2sg}-pound-\textsc{fv}\\
\glt `They might (eventually) knock you over.' [overheard]
\end{exe}

These readings of the indefinite future are probably the reason that \citet{NurseD1979} considers this construction a \lq\lq distant future''. Rather than a question of temporal remoteness, this interpretation apparently derives from the strongly modal semantics of the construction. To begin with, even in predictions of a temporally very distant time, the collocation of future \isi{proclitic} \textit{aa}=\is{future!future aa=} plus the \isi{simple present} (\sectref{ProcliticAa}) is normally used:

\begin{exe}
\ex \gll lɪnga fi-kɪnd-ile ɪ-fy-ɪnja f-ingi ɪ-fi fi-kw-is-a \textbf{aa}=\textbf{bi}-\textbf{kʊ}-\textbf{mal}-\textbf{a} ɪ-n-jɪla j-aa kʊ-bʊʊk-a kʊ-Tʊkʊjʊ\\
if/when 8-pass-\textsc{pfv} \textsc{aug}-8-year 8-many \textsc{aug}-\textsc{prox.8} 8-\textsc{prs}-come-\textsc{fv} \textsc{fut}=2-\textsc{prs}-finish-\textsc{fv} \textsc{aug}-9-path 9-\textsc{assoc} 15-go-\textsc{fv} 17-T.\\
\glt \lq In many years, they will finish the road to Tukuyu.' [ET]
\end{exe}

Furthermore, the indefinite future itself can be combined with the future \isi{proclitic} \textit{aa}=\is{future!future aa=} (\ref{exIndefiniteFutureAa}). In fact, when checking in elicitation how far this construction can be combined with definite temporal adverbials, speakers nearly always added \textit{aa}= when repeating the constructed examples.

\begin{exe}
\ex \label{exIndefiniteFutureAa} \gll aa=tw-isakʊ-fik-a kɪ-laabo\\
\textsc{fut}=\textsc{1pl}-\textsc{indef.fut}-arrive-\textsc{fv} 7-tomorrow\\
\glt \lq We will probably arrive tomorrow.' [ET]
\end{exe}

The indefinite future can further be used to express epistemic modality without future time reference (\ref{exIsaFutEpistemic1}, \ref{exIsaFutEpistemic2}). As \citet{BybeePagliucaPerkins1991} point out, epistemic uses other than future prediction are typical of old future forms. Indeed, as (\ref{exIsaFutEpistemic3}) shows, this use was already available in the nineteen-forties.

\begin{exe}
\ex \label{exIsaFutEpistemic1}
Context: You are asked where your brother is. You assume he is playing football.\\
\gll \textbf{isakʊ}-\textbf{kin}-\textbf{a} ʊ-m-pɪla\\
1.\textsc{indef.fut}-play-\textsc{fv} \textsc{aug}-3-ball\\
\glt `He will be [=presumably he is] playing football.' [ET] %auch mal gucken, ob -aga geht

\clearpage

\ex \label{exIsaFutEpistemic2}
Context: The cows are mooing loudly. You are asked why they are making so much noise. You assume they are hungry.
\gll \textbf{s}-\textbf{isakʊ}-\textbf{j}-\textbf{a} n=ɪ-n-jala\\
10-\textsc{indef.fut}-be(come)-\textsc{fv} \textsc{com}=\textsc{aug}-9-hunger\\
\glt `They will be [=presumably they are] hungry.' [ET]
\ex \label{exIsaFutEpistemic3}
\gll ʊ-mw-ana i-kʊ-lɪl-a n-nyumba. \textbf{isakʊ}-\textbf{bunguluk}-\textbf{ɪl}-\textbf{a} m-m-ooto\\
\textsc{aug}-1-child 1-\textsc{prs}-cry-\textsc{fv} 18-house(9) 1.\textsc{indef.fut}-toss\_and\_turn-\textsc{appl}-\textsc{fv} 18-3-fire\\
\glt \lq The chid is crying in the house. It'll have [=presumably it has] fallen into the fire.' (Busse 1949: 199; orthography adapted)
\end{exe}

Lastly, imperfective\is{aspect!imperfective} -\textit{aga} adds a continuous or \is{aspect!habitual} reading:
\begin{exe}
\ex \gll n-isak-ʊʊl-aga ɪ-fi\\
\textsc{1sg}-\textsc{indef.fut}-buy-\textsc{ipfv} \textsc{aug}-\textsc{prox.8}\\
\glt \lq I will (probably or eventually) buy these things (habitually).' [ET]
\ex \gll isakʊ-jeng-aga papaa$\sim$pa\\
1.\textsc{indef.fut}-build-\textsc{ipfv} \textsc{redupl}$\sim$\textsc{prox.16}\\
\glt \lq He will (probably or eventually) be building right here.' [ET]
\end{exe}

\subsection{A diachronic perspective}\label{isaFutDiachronic}\is{grammaticalization|(} In the introduction to this section it was commented that the indefinite future constitutes an advanced stage of grammaticalization of \textit{isa} `come' as an \isi{auxiliary} in the simple present.\is{simple present} While the phonetic attrition is self-evident, the construction's meaning and use suggest a more complex chain of developments. As discussed in \sectref{isaAspectualizer}, the de-ventive auxiliary \textit{isa} has an ingressive reading and in its \isi{futurate} use contrasts with the de-itive prospective/movement construction (\sectref{Prospectivekwa}).\is{futurate}\is{aspect!prospective} Only the latter of the two allows a component of intentionality. Now, in the first grammatical sketches of Nyakyusa (\citealt{SchumannK1899}; \citealt{EndemannC1914}), the indefinite future construction is listed as the main future tense. Correspondingly it is relatively frequent in the text collections from that time (\citealt{BergerP1933}; \citealt{BusseJ1942}; \citeyear{BusseJ1949}), where it is found with future predictions that do not necessarily involve an ingressive component. One can thus assume a semantic generalization from an assertion about reaching a certain condition in the future towards a prediction about a future state-of-affairs. This is precisely what has been documented for de-ventive futures in Swedish \citep{ChristensenL1997} and Rhaeto-Romance \citep{EbneterT1973}. Such a grammaticalization path is also suggested by \citet{TraugottE1978}.

While at the turn of the twentieth century the indefinite future apparently constituted a core construction in Nyakyusa's TMA system, the above discussion has shown that this does not hold for the present-day language. Note also that \citeauthor{LusekeloA2007} (\citeyear{LusekeloA2007}; \citeyear{LusekeloA2013}), in his discussion of tense and aspect in Nyakyusa, does not list the indefinite future at all, but states that the collocation of the future clitic \textit{aa}=\is{future!future aa=} plus the \isi{simple present} (\sectref{ProcliticAa}) is the only future tense\is{tense!future} in this language. \citet{NurseD1979}, as noted above, considers the construction in question a \lq\lq distant future'' in contrast to the \lq\lq intermediate future'' expressed through \mbox{\textit{aa}=} plus simple present.\is{simple present}

The history of the indefinite future as attested in the first descriptions of Nyakyusa and its usage in earlier text collections, its strongly modal\is{modality} flavour in the present-day language plus the fact that it allows for a non-temporal epistemic reading, as well as its distribution and combinatory possibilities, all point towards an old future construction that has mainly been displaced into the modal dimension.\is{modality} This most likely went along with the future proclitic \textit{aa}=\is{future!future aa=} (\sectref{ProcliticAa}), especially in collocation with the simple present, occupying the former territory of the construction.\footnote{Note also that some language assistants commented on the use of the indefinite future in typical prediction contexts: ``Maybe you would use that in the written language. But in speaking, we don't say it like that''. The only more widely known written materials in Nyakyusa are translations of the Bible, which are based on the variety described by \citet{SchumannK1899} and \citet{EndemannC1914}.}\is{grammaticalization|)}
\is{future!indefinite future|)}\is{modality|)}\is{tense!future|)}
\section{Prospective/Inceptive \textit{ja pa}-INF}\label{ProspectiveKujapa}%fertig
\is{futurate|(}\is{aspect!prospective|(}
This construction consists of the \isi{copula} verb \textit{ja} \lq be(come)' plus an \isi{infinitive} marked for \isi{locative} noun class 16.

\begin{exe}
\ex \textit{tʊkʊja pakʊjoba} \lq we are about to speak'
\end{exe}

Formally this construction constitutes the inceptive counterpart to the periphrastic progressive\is{aspect!progressive} (\sectref{Progressive}). As for its semantics, it construes a future eventuality as very near or imminent to a point of reference, by default the time of speech. This often goes along with a sense of intention. Unlike the de-itive construction described in \sectref{Prospectivekwa}, this construction does not have a possible reading of physical motion.\is{motion} The following examples illustrate the use of this construction in the present tense.\is{tense!present}

\begin{exe}
\ex \label{exPRSpakuProspectiveNearIntention}
\gll ʊ-n-kiikʊlʊ n-ummw-ag-ile, \textbf{tʊ}-\textbf{kʊ}-\textbf{j}-\textbf{a} \textbf{pa}-\textbf{kw}-\textbf{eg}-\textbf{an}-\textbf{a} kɪfuki\\
\textsc{aug}-1-woman \textsc{1sg}-1-find-\textsc{pfv} \textsc{1pl}-\textsc{prs}-be(come)-\textsc{fv} 16-15-marry-\textsc{recp}-\textsc{fv} near\\
\glt `I've found a woman, we'll marry soon.' [Hare and Spider]
\ex \label{exPRSpakuProspectiveIntention}
\gll bo \textbf{kʊ}-\textbf{j}-\textbf{a} \textbf{pa}-\textbf{kʊ}-\textbf{nw}-\textbf{a} ʊ-n-kota ʊ-gʊ taasi ʊ-ka-bʊʊk-ege n=ʊ-lʊ-bʊnjʊ m-ma-tengele\\
as \textsc{2sg.prs}-be(come)-\textsc{fv} 16-15-drink-\textsc{fv} \textsc{aug}-3-medicine \textsc{aug}-\textsc{prox}.3 first \textsc{2sg}-\textsc{itv}-go-\textsc{ipfv.subj} \textsc{com}=\textsc{aug}-11-morning 18-6-bush\\
\glt `When you plan to drink this portion, first go into the bush in the morning.' [Mfyage turns into a lion]\
\ex %damit BSP ohne intention
\gll ɪɪ-fula \textbf{jɪ}-\textbf{kʊ}-\textbf{j}-\textbf{a} \textbf{pa}-\textbf{kʊ}-\textbf{tim}-\textbf{a}\\
\textsc{aug}-rain(9) 9-\textsc{prs}-be(come)-\textsc{fv} 16-15-rain-\textsc{fv}\\
\glt \lq It is about to rain.' [ET]  
\end{exe}

While the prospective/movement construction discussed in \sectref{Prospectivekwa} only has a prospective reading when used in the simple present,\is{simple present} the prospective/inceptive construction can be used in the past tense:\is{tense!past}
\begin{exe}
\ex \gll bo m-fik-ile pa-ka-aja, ʊ-n-nuguna gw-angʊ \textbf{a}-\textbf{a}-\textbf{j}-\textbf{aga} \textbf{pa}-\textbf{kʊ}-\textbf{sook}-\textbf{a}=\textbf{po}\\
as \textsc{1sg}-arrive-\textsc{pfv} 16-12-homestead \textsc{aug}-1-younger\_sibling 1-\textsc{poss.1sg} 1-\textsc{pst}-be(come)-\textsc{ipfv} 16-15-leave-\textsc{fv}=16\\
\glt \lq When I arrived at home my younger brother was about to leave.' [ET]
\end{exe}

\is{futurate|)}\is{aspect!prospective|)}